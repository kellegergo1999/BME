\documentclass[12pt]{article}
\usepackage{blindtext}
\usepackage[top=25mm, bottom=25mm, left=20mm, right=20mm]{geometry}
\usepackage{t1enc}
\usepackage{fancyhdr}
\usepackage[english]{babel}
\usepackage{lastpage}
\usepackage{amsmath}
\usepackage{amssymb,physics}
\usepackage{amsthm}
\usepackage{empheq}
\usepackage{graphicx}
\usepackage{wrapfig}
\usepackage{tikz}
\usepackage{gensymb}
\usepackage{pdfpages}
\usepackage{hyperref}
\usepackage{multirow}
\usepackage{pgfplots}
\usepackage{float}
\usepackage{caption}
\usepackage{subcaption}
\pgfplotsset{compat = newest}
\usepackage{listings}
\usepackage[makeroom]{cancel}
\usepackage{xcolor}

\definecolor{codegreen}{rgb}{0,0.6,0}
\definecolor{codegray}{rgb}{0.5,0.5,0.5}
\definecolor{codepurple}{rgb}{0.58,0,0.82}
\definecolor{backcolour}{rgb}{0.9059,0.902,0.9137}
\hypersetup{
    colorlinks=true,
    linkcolor=black,
    filecolor=magenta,      
    urlcolor=cyan,
}

\urlstyle{same}
\usepackage{graphicx}
\graphicspath{ {./images/} }
\usepackage{pict2e}
\usepackage{blindtext}
\usepackage{tikz}
\newcommand*\circledB{%
  \tikz[baseline=(char.base)]{%
    \node[shape=circle,draw,inner sep=1pt] (char) {b};%
  }%
}
% Table float box with bottom caption, box width adjusted to content


\newcommand*\circled[1]{%
  \tikz[baseline=(char.base)]{%
    \node[shape=circle,draw,inner sep=1pt] (char) {#1};%
  }%
}


\DeclareRobustCommand{\slashcirc}{{\mathpalette\doslashcirc\relax}}

\usepackage{graphicx,array}
\newcolumntype{C}[1]{>{\centering\let\newline\\\arraybackslash\hspace{130pt}}m{#1}}
\newcolumntype{L}[1]{>{\raggedright\let\newline\\\arraybackslash\hspace{130pt}}m{#1}}

\makeatother

\usepackage{amsmath}
\newcommand\aug{\fboxsep=-\fboxrule\!\!\!\fbox{\strut}\!\!\!}
\usepackage{wasysym}
\usepackage{tikz}
\usepackage{xcolor}
%\def\checkmark{\tikz\fill[scale=0.4](0,.35) -- (.25,0) -- (1,.7) -- (.25,.15) -- cycle;} 
\newcommand*\widefbox[1]{\fbox{\hspace{2em}#1\hspace{2em}}}
\newcommand{\fakesection}[1]{%
% Increase section counter
  % Add section mark (header)
  \addcontentsline{toc}{section}{\protect\numberline{\thesection}#1}% Add section to ToC
  % Add more content here, if needed.
}
\renewcommand{\headrulewidth}{0.5pt}
\renewcommand{\footrulewidth}{0.5pt}
\newcommand{\horrule}[1]{\rule{\linewidth}{#1}}
\pagestyle{fancy}
\newcommand\invisiblesection[1]{%
  \refstepcounter{section}%
  \addcontentsline{toc}{section}{\protect\numberline{}Feladatkitűzés}
  \sectionmark{1}}
\newcommand{\NEV}{Gergő Kelle}
\newcommand{\NEPTUN}{GBBNUL}
\newcommand{\TANTARGYKOD}{BMEGEMMNWFE}
\newcommand{\HFCIME}{Finite Element Analysis \\ III. Homework}
\newcommand{\HFFEJLECCIME}{Finite Element Analysis}
\title{\centering \includegraphics[width=10 cm , height=2.8cm]{bme_logo_nagy}
\bigskip

\normalfont \normalsize \textsc{\centering Budapest University of Technology and Economics
\\ Faculty of Mechanical Engineering} \\ [12pt] \horrule{0.5pt} \\[0.4cm] \huge \HFCIME{} \\ \horrule{2pt} \\[0.5cm]}
\lhead{\TANTARGYKOD, \HFFEJLECCIME{}}
\rfoot{\thepage\ / \pageref{LastPage}}
\cfoot{}
\rhead{\NEV{}, \NEPTUN{}}
\author{\NEV{}}
\lstdefinestyle{mystyle}{
    backgroundcolor=\color{backcolour},   
    commentstyle=\color{codegreen},
    keywordstyle=\color{magenta},
    numberstyle=\tiny\color{codegray},
    stringstyle=\color{codepurple},
    basicstyle=\ttfamily\footnotesize,
    breakatwhitespace=false,         
    breaklines=true,                 
    captionpos=b,                    
    keepspaces=true,                 
    numbers=left,                    
    numbersep=5pt,                  
    showspaces=false,                
    showstringspaces=false,
    showtabs=false,                  
    tabsize=2
}

\lstset{style=mystyle}

\begin{document}
\includepdf[pages=1]{Finite_Element_Analysis_HW_03_kitoltott.pdf}
\pagebreak
\maketitle
\pagebreak

\pagebreak
\tableofcontents
\newpage
\section{Equation of motion on a multi-DoF system}
Create the \textit{FE} model of the structure in \textit{MAXIMA} using \textit{BEAM1D} elements and rigid disks using the minimally required number of elements! Determine the structural matrices of the structure and calculate the first four eigenfrequencies of the system!\\
\horrule{0.4pt}

\subsection{Parameters and discretization}

\begin{table}[hbt]
    \centering
    \begin{minipage}{0.45\textwidth}
        \centering
        \label{tab:data}
        \caption{Parameters for Homework ID 121}
        \begin{tabular}{ccc}
            Parameter & Value   & Dimension            \\ \hline
            $L_1$     & 0.28    & m                    \\
            $D_1$     & 0.35    & m                    \\
            $\omega$  & 370     & $\frac{\text{rad}}{\text{s}}$      \\
            $d$       & 0.06    & m                    \\
            $F_{0}$   & 100     & N                    \\
            $M_{0}$   & 10      & Nm                   \\
            $E$   & 210      & GPa                   \\
            $\rho$   & 7860      & $\frac{\text{kg}}{\text{m}^3}$                   \\[0.1cm] \hline
            Scheme:    & \multicolumn{2}{c}{Constant acceleration}
        \end{tabular}
    \end{minipage}%
    \begin{minipage}{0.55\textwidth}
        \centering
        \hspace{0.2cm}
        \includegraphics[width=0.95\textwidth]{abra.pdf}
        \captionof{figure}{Your Figure Caption}
        \label{fig:your-figure-label}
    \end{minipage}
\end{table}
Before any further calculations, some base parameters need to be determined.

\begin{gather}
D_2 = 0.8~ D_1 = 0.28 ~ \text{[m]}\\[0.1cm]
D_3 = 1.3~ D_1 = 0.455 ~ \text{[m]} \\[0.1cm]
t_1 = \dfrac{D_1}{10} = 0.035 ~ \text{[m]}\\[0.1cm]
t_2 = \dfrac{D_1}{15}  = 0.0187 ~ \text{[m]}\\[0.1cm]
t_3 = \dfrac{D_1}{12} = 0.0379 ~ \text{[m]}
\end{gather}

\noindent Based on Table \ref{tab:data} values and the determined parameters above the disk's mass and the moment of inertia can be determined.

\begin{gather}
m_1 = \pi ~ \rho ~ t_{1} \dfrac{\left(D_{1}^{2} - d^{2}\right)}{4} = 25.67 ~ \text{[kg]}\\[0.1cm]
m_2 = \pi ~ \rho ~ t_{2} \dfrac{\left(D_{2}^{2} - d^{2}\right)}{4} = 8.62 ~ \text{[kg]}\\[0.1cm]
m_3 = \pi ~ \rho ~ t_{3} \dfrac{\left(D_{3}^{2} - d^{2}\right)}{4} = 47.62 ~ \text{[kg]}
\end{gather}
\begin{gather}
\theta_1 = \dfrac{1}{4} m_1 \left( \frac{D_1^2}{4} + \dfrac{d^2}{4} \right) + \dfrac{1}{12} m_1 ~ t_1^2 = 0.2051 ~ \text{[kgm]}^2\\[0.1cm]
\theta_2 = \dfrac{1}{4} m_2 \left( \frac{D_2^2}{4} + \dfrac{d^2}{4} \right) + \dfrac{1}{12} m_2 ~ t_2^2 = 0.0444 ~ \text{[kgm]}^2\\[0.1cm]
\theta_3 = \dfrac{1}{4} m_3 \left( \frac{D_3^2}{4} + \dfrac{d^2}{4} \right) + \dfrac{1}{12} m_2 ~ t_3^2 = 0.6325 ~ \text{[kgm]}^2
\end{gather}
For the mass and stiffness matrices the area and second moment of inertia has to be determined.
\begin{gather}
A = \dfrac{d ^2  \pi}{4} = 0.0028274 ~ \text{[m]}^2 \\
I_z = \dfrac{d^4  \pi}{64} = 6.3617251 \cdot 10^{-7} ~ \text{[m]}^4
\end{gather} 
Due to the circular shape of the beam's cross section remaining constant along the X-axis, the area and second moment of inertia remains consistent throughout the beam. The
information about discretization showed in Table \ref{tab:disc}.

\begin{table}[hbt]
\label{tab:disc}
\captionsetup{justification=centering,position=top} % set caption above table and centered
\caption{Discretization} % common caption for both tables
\begin{minipage}[t]{0.45\linewidth}
\centering
\captionsetup{justification=centering} % set subcaption centered
\begin{tabular}{ c c c }
 Node & x $\left[ \text{mm} \right]$ & y $\left[ \text{mm} \right]$ \\
 \hline
 1 & 0 & 0 \\
 2 & 280 & 0 \\
 3 & 560 & 0 \\
 4 & 840 & 0 \\
 5 & 1120 & 0 \\
\end{tabular}
\subcaption{Node coordinates} % subcaption for first table
\label{tab:node-coordinates}
\end{minipage}
\quad
\begin{minipage}[t]{0.45\linewidth}
\centering
\captionsetup{justification=centering} % set subcaption centered
\begin{tabular}{ c c c }
 Element & Local node 1 & Local node 2 \\
 \hline
 1 & 1 & 2 \\
 2 & 2 & 3 \\
 3 & 3 & 4 \\
 4 & 4 & 5 \\
\end{tabular}
\vspace{0.24 cm}
\subcaption{Element connectivity} % subcaption for second table
\label{tab:element-connectivity}
\end{minipage}
\end{table}
\vspace{-0.3cm}
\subsection{Determination of the first four eigenfrequencies}

\begin{table}[H]
\centering
\caption{Main equations}
\label{tab:important_equations}
\renewcommand{\arraystretch}{1.2}
\begin{tabular}{lc}
\textbf{Parameter} & \textbf{Equation} \\
\hline
\textbf{Consistent mass matrix} &  \parbox{10cm}{\begin{align} \textbf{M}_e = \frac{Al\rho}{420} \begin{bmatrix}
156 & 22l & 54 & -13l \\
22l & 4l^2 & 13l & -3l^2 \\
54 & 13l & 156 & -22l \\
-13l & -3l^2 & -22l & 4l^2
\end{bmatrix}  \end{align}} \\

\textbf{Material stiffness matrix} & \parbox{10cm}{\begin{align} \textbf{K}_e = \frac{I_z El^3}{12} \begin{bmatrix}
12 & 6l & -12 & 6l \\
6l & 4l^2 & -6l & 2l^2 \\
-12 & -6l & 12 & -6l \\
6l & 2l^2 & -6l & 4l^2
\end{bmatrix}  \end{align}} \\

\end{tabular}
\end{table}
The stiffness and mass matrices of all four discretized elements are identical due to the beam's uniform cross-section and isotropic material properties. Thus the global matrices are obtained by summing the identical mass or stiffness matrices with regard to it's DoF. This way we get 10 $\cross$ 10 matrices that has to be condensed based on the boundary conditions of $v_2 = 0$ and $v_4= 0$ resulting the ellimination of the 3rd and 7th rows and columns of the structural matrices. Condensed
matrices will be 8 $\cross$ 8 matrices.\newline
Another thing to point out is the mass matrix for the whole structure, because we have to consider the disk. Thus before the structural mass condensation, we should substitute the beam's parameters into \textbf{M} then calculate the global structural mass matrix of the beam $\textbf{M}_{beam}$ and add the disk's mass matrix Mdisk leaving us with the following global mass matrix:
\begin{equation}
\textbf{M} = \textbf{M}_{beam} + \textbf{M}_{disk}
\end{equation}
\begin{equation*}
	\hspace{-6 cm}
	\begin{aligned}
	\text{Where} \quad 
	\begin{array}{|l@{} c l}
M_{disk} &~=& \begin{bmatrix}
m_1 & 0 & 0 & 0 & 0 & 0 & 0 & 0 & 0 & 0 \\
0 & \theta_1 & 0 & 0 & 0 & 0 & 0 & 0 & 0 & 0 \\
0 & 0 & 0 & 0 & 0 & 0 & 0 & 0 & 0 & 0 \\
0 & 0 & 0 & 0 & 0 & 0 & 0 & 0 & 0 & 0 \\
0 & 0 & 0 & 0 & m_2 & 0 & 0 & 0 & 0 & 0 \\
0 & 0 & 0 & 0 & 0 & \theta_2 & 0 & 0 & 0 & 0 \\
0 & 0 & 0 & 0 & 0 & 0 & 0 & 0 & 0 & 0 \\
0 & 0 & 0 & 0 & 0 & 0 & 0 & 0 & 0 & 0 \\
0 & 0 & 0 & 0 & 0 & 0 & 0 & 0 & m_3 & 0 \\
0 & 0 & 0 & 0 & 0 & 0 & 0 & 0 & 0 & \theta_3
\end{bmatrix}
	\end{array}
	\end{aligned}
\end{equation*}
After the condensation we can write out the equation of motion of the system as
\begin{equation}
\textbf{M}_c~\ddot{\textbf{u}}_c + \textbf{K}_c ~ \textbf{u}_c = \textbf{F}_c
\end{equation}
To calculate the natural angular freqency we can use the following formula
\begin{equation}
\det \left( -\omega_n^2 \textbf{M}_c + \textbf{K}_c \right)=0.
\end{equation}
By calculating the first four solution for $\omega_n$ we get the following results:
\begin{table}[hbt]
\label{tab:disc} 
\centering
\caption{Eigenfreqencies} % common caption for both tables
\label{tab:freq_sol}
\begin{tabular}{ c c c }
 i & $\omega_{n,i} \left[ \frac{\text{rad}}{\text{s}} \right]$ &  $ f_{n,i} \left[ \text{Hz} \right]$ \\[0.2cm]
 \hline
 1 & 300.08 & 47.76 \\
 2 & 484.22 & 77.07 \\
 3 & 1669.45 & 265.70 \\
 4 & 2429.42 & 386.65 \\
\end{tabular}
\end{table}

\begin{equation*}
	\hspace{-14 cm}
	\begin{aligned}
	\text{Where} \quad 
	\begin{array}{|l@{} c l}
f_n = \dfrac{\omega_n}{2 \pi}
	\end{array}
	\end{aligned}
\end{equation*}

\section{The first eigenfrequency using Rayleigh's Principle}
Apply the Rayleigh-quotient, determine the first eigenfrequency approximately and verify the finite element model!\\
\horrule{0.4pt}


To determine the first eigenfrequency of the beam using the Rayleigh-quotient method, we begin by approximating the mode shape of the beam as a polynomial function, which satisfies the boundary conditions. Since the Rayleigh method is based on kinematic conditions, we can only use boundary constraints that restrict the kinematics of the system.
At $x = L_1$, the beam has a pinned support, so the deformation in the y direction should be zero.
\begin{equation}
Y(L_1) = 0
\end{equation}
Similarly, at $x = 3L_1$, the beam has a pinned-roller support, so the deformation in the y direction should be zero.
\begin{equation}
Y(3L_1) = 0
\end{equation}

Based on these two boundary conditions, we can estimate the mode shape function as a second-order polynomial of the form 
\begin{equation}
Y(x) = x^2 + a_1x + a_0,
\end{equation}

where $a_i$, $i = 0, 1$ are constant parameters. We also require the first and second derivatives of $Y(x)$ with respect to $x$, which are 
\begin{equation}
Y'(x) = 2x + a_1
\end{equation}
and 
\begin{equation}
Y''(x) = 2,
\end{equation} 
respectively.

By substituting the boundary conditions into the polynomial, we can obtain a system of equations for the constants $a_0$ and $a_1$. Solving these equations yields 
\begin{equation}
a_0 = 3L_1^2 = 0.2352 ~[m]
\end{equation}
and 
\begin{equation}
a_1 = -4L_1 =-1.12 ~[1].
\end{equation}
Using these constants, we can calculate the approximate mode shape function of the beam, which is 
\begin{equation}
Y(x) = x^2 -1.12x + 0.2352.
\end{equation}

\begin{figure}[H]
\centering
\includegraphics[width=0.7\linewidth]{plot_jobb.pdf}
\caption{The $Y(x)$ approximate modeshape function}
\end{figure}


The total potential energy in this case is described by the following equation:
\begin{equation}
U = U_{shaft} = \frac{1}{2} I_z ~ E ~ \cos^2 \left( \omega_n ~ t \right) ~ \int_0^{4L_1}  Y''^2 (x)~\text{d}x
\end{equation}
The kinetic energy can be calculated with the following integral:

\begin{equation}
T = T_{\text{shaft}} + T_{\text{disks}} = \frac{1}{2} \left( \rho ~ A ~\int_0^{4L_1}  Y(x) ~\text{d}x + m_1 ~ Y^2(0)+ \theta_1 ~ Y'^2(0)+ \right.
\end{equation}
\vspace{-0.3cm}
\begin{equation*}
\left. +m_2 ~ Y^2(2L_1) + \theta_2 Y'^2(2L_1)+ m_3 Y^2 (4L_1) + \theta_3 Y'^2(L)\right) \omega_n^2 ~ \sin^2\left(\omega_n ~ t \right)
\end{equation*}
Based on the conservation of energy, Raylaigh’s principle states that the maximum of the potential and kinetic energies are equal:
\begin{equation}
T_{max} = U_{max}
\end{equation}
By considering the maximum values of both the $\sin^2(\omega_n~ t)$ and $\cos^2(\omega_n ~ t)$ functions, which are equal to 1, we can derive the following formula to approximate the first eigenfrequency of the system.
\begin{equation}
\resizebox{0.9\linewidth}{!}{$\omega_{n,1}^R \leq \sqrt{\frac{I_z E \int_0^{4L_1} Y''^2(x) \, dx}{\rho A \int_0^{4L_1} Y(x) \, dx + m_1 Y^2(0) + \theta_1 Y'^2(0) + m_2 Y^2(2L_1) + \theta_2 Y'^2(2L_1) + m_3 Y^2(4L_1) + \theta_3 Y'^2(4L_1)}}$}
\end{equation}
By calculating this equation we get the following results:
\begin{equation}
\omega_{n,1}^R \leq 333.12 \left[ \frac{\text{rad}}{\text{s}} \right]
\end{equation}
\begin{equation}
f_{n,1}^R = \frac{2\pi}{\omega_{n,1}} = 53.02 ~  \left[ \text{Hz} \right]
\end{equation}
By applying the Rayleigh-quotient method, we've obtained an expression for the first eigenfrequency of the beam. Once we have the first eigenfrequency, we can verify the finite element model by comparing the calculated eigenfrequency with the one obtained from the finite element analysis. If they match, it confirms that the FE model accurately represents the behavior of the beam under bending vibrations.
\begin{equation}
\dfrac{f_{n,1}^R - f_{n,1} }{f_{n,1}} \cdot 100 = 11.01 ~ [\%]
\end{equation}
The relative error in our approximation may not be very low, but it's important to consider that we used a second-order polynomial as the estimating mode shape function. In summary, we employed the Rayleigh method to estimate the first natural frequency and corresponding mode shape of the system. This method provides an upper estimate for the first eigenfrequency. Based on our results, we can conclude that the finite element model we applied is valid.


\section{Response of the system using direct time integration}
Determine the response of the system by direct time integration! Determine the required time step size! Apply the assigned method and plot the transverse displacement and the angle of rotation at cross section $K$! Apply 2500-3000 time steps!\\
\horrule{0.4pt}
The direct time integration method can be employed to determine the system's time response under any given excitation. In my assigned task, I utilized the constant acceleration method, also known as the Newmark algorithm. The time integration scheme involved utilizing the $\alpha$ and $\gamma$ parameters, as follows:

\begin{gather}
\alpha = \dfrac{1}{4} \\[0.1cm]
\gamma =  \dfrac{1}{2}
\end{gather}
The constant acceleration scheme is unconditionally stable due to the $\alpha$ and $\gamma$ parameters. A sufficiently small time step ($\Delta t$) is still needed for accurate system representation. Since we focused on the first four eigenfrequencies, I will be using $\omega_{n,4} = 2429.42 ~ \left[\frac{\text{rad}}{\text{s}}\right]$ to determine the timestep size. The corresponding period for this frequency is:
\begin{equation}
T_{n,4} = \dfrac{2 \pi}{\omega_{n,4}} = 2.5863 \cdot 10^{-3} ~ \text{[s]}
\end{equation}
For the timestep size I will be using one tenth of it 
\begin{equation}
\Delta t = \dfrac{T_{n,4}}{10} = 2.5863 \cdot 10^{-4} ~ \text{[s]}.
\end{equation}

The discrete time scale can be determined as
\begin{equation}
t = n ~ \Delta t
\end{equation}
Where n is the number of steps which is choose to be $n = 3000$, meaning that $t$ is the time of the simulation. The vector of nodal excitation at the $n$th timestep becomes
\begin{equation}
\textbf{F}_n^T = \begin{bmatrix}
0 & 0 & M_0 \cos \left( \omega_n ~n ~ \Delta t \right) & 0 & 0 & 0 & 0 & 0
\end{bmatrix} ~ [\text{SI}]
\end{equation}
The Newmark algorithm provides the following displacement and velocity vectors at the (n + 1)th time step:
\begin{gather}
\textbf{u}_{n+1} = \textbf{u}_n + \Delta t ~ \dot{\textbf{u}}_n + \frac{1}{2} \Delta t ~ \ddot{\textbf{u}}_{n+\alpha} \\
\dot{\textbf{u}}_{n+1} = \dot{\textbf{u}}_n+ \Delta t ~ \ddot{\textbf{u}}_{n+ \gamma}
\end{gather}
The displacement for the constant acceleration scheme can be determined by analyzing the equation of motion at the n and $n+1$ timesteps
\begin{equation}
\textbf{u}_{n+1} = \left( \dfrac{4}{\Delta t^2} \textbf{M }+ \textbf{K} \right) ^{-1} \left[ \textbf{F}_{n+1} + \textbf{M} \left( \ddot{\textbf{u}}_n + \dfrac{4}{\Delta t} \dot{\textbf{u}}_n + \dfrac{4}{\Delta t^2} \textbf{u}_n \right) \right].
\end{equation}
\begin{gather}
\dot{\textbf{u}}_{n+1} = \dfrac{2}{\Delta t} \left( \textbf{u}_{n+1} - \textbf{u}_n\right) - \dot{\textbf{u}}_n \\
\ddot{\textbf{u}}_{n+1} = \dfrac{4}{\Delta t^2} \left( \textbf{u}_{n+1}- \textbf{u}_n \right) - \dfrac{4}{\Delta t} \dot{\textbf{u}}_n- \ddot{\textbf{u}}_n
\end{gather}
One notable benefit of any form of the Newmark iteration scheme is its inherent self-starting capability. By providing only the initial values for nodal displacement, velocity, and acceleration, the algorithm can initiate without requiring any additional information. In this particular scenario, we assume that the nodal displacement and velocities at the initial step are both zero, denoted as
\begin{equation}
\textbf{u}_0 = \textbf{0}
\end{equation}
and
\begin{equation}
\dot{\textbf{u}}_0 = \textbf{0},
\end{equation}
respectively. The initial acceleration can be determined from the equation of motion at $t = 0 (n = 0)$:
\begin{equation}
\textbf{M} ~ \ddot{\textbf{u}}_0 + \textbf{K} ~ \textbf{u}_0 = \textbf{F}_0
\end{equation}
$$ \downarrow $$
\begin{equation}
\ddot{\textbf{u}}_0 = \textbf{M}^{-1} \left( -\textbf{K} ~ \textbf{u}_0 + \textbf{F}_0 \right)
\end{equation}
The numerical values for the initial acceleration is
\begin{equation}
\ddot{\textbf{u}}_0 = \begin{bmatrix}
2.1569\\
18.1791\\
1150.67\\
-4.6828 \\
74.6986\\
0.842038\\
-0.00090136\\
0.00447603
\end{bmatrix} ~ [\text{SI}].
\end{equation}

\begin{figure}[H]
\centering
    \includegraphics[width=0.6\linewidth]{plotocska_jo2.pdf}
    \caption{Timeresponse for $v_3$}
    \label{fig:plot1}
\end{figure}

\begin{figure}[H]
\centering
    \includegraphics[width=0.6\linewidth]{plotocska_jo1.pdf}
    \caption{Timeresponse for $\varphi_3$}
    \label{fig:plot2}
  \label{fig:combined}
\end{figure}

\begin{figure}[H]
\centering
\includegraphics[width=0.6\linewidth]{plotocska_jo3.pdf}
\caption{Timeresponse for the $v_3$ and $\varphi_3$ DoFs}
\end{figure}
\end{document}
