\documentclass[12pt]{article}
\usepackage{blindtext}
\usepackage[top=25mm, bottom=25mm, left=20mm, right=20mm]{geometry}
\usepackage{t1enc}
\usepackage{fancyhdr}
\usepackage[english]{babel}
\usepackage{lastpage}
\usepackage{amsmath}
\usepackage{amssymb,physics}
\usepackage{amsthm}
\usepackage{empheq}
\usepackage{graphicx}
\usepackage{wrapfig}
\usepackage{tikz}
\usepackage{gensymb}
\usepackage{pdfpages}
\usepackage{hyperref}
\usepackage{multirow}
\usepackage{pgfplots}
\usepackage{float}
\usepackage{caption}
\usepackage{subcaption}
\pgfplotsset{compat = newest}
\usepackage{listings}
\usepackage[makeroom]{cancel}
\usepackage{xcolor}

\definecolor{codegreen}{rgb}{0,0.6,0}
\definecolor{codegray}{rgb}{0.5,0.5,0.5}
\definecolor{codepurple}{rgb}{0.58,0,0.82}
\definecolor{backcolour}{rgb}{0.9059,0.902,0.9137}
\hypersetup{
    colorlinks=true,
    linkcolor=black,
    filecolor=magenta,      
    urlcolor=cyan,
}

\urlstyle{same}
\usepackage{graphicx}
\graphicspath{ {./images/} }
\usepackage{pict2e}
\usepackage{blindtext}
\usepackage{tikz}
\newcommand*\circledB{%
  \tikz[baseline=(char.base)]{%
    \node[shape=circle,draw,inner sep=1pt] (char) {b};%
  }%
}
% Table float box with bottom caption, box width adjusted to content


\newcommand*\circled[1]{%
  \tikz[baseline=(char.base)]{%
    \node[shape=circle,draw,inner sep=1pt] (char) {#1};%
  }%
}


\DeclareRobustCommand{\slashcirc}{{\mathpalette\doslashcirc\relax}}

\usepackage{graphicx,array}
\newcolumntype{C}[1]{>{\centering\let\newline\\\arraybackslash\hspace{130pt}}m{#1}}
\newcolumntype{L}[1]{>{\raggedright\let\newline\\\arraybackslash\hspace{130pt}}m{#1}}

\makeatother

\usepackage{amsmath}
\newcommand\aug{\fboxsep=-\fboxrule\!\!\!\fbox{\strut}\!\!\!}
\usepackage{wasysym}
\usepackage{tikz}
\usepackage{xcolor}
%\def\checkmark{\tikz\fill[scale=0.4](0,.35) -- (.25,0) -- (1,.7) -- (.25,.15) -- cycle;} 
\newcommand*\widefbox[1]{\fbox{\hspace{2em}#1\hspace{2em}}}
\newcommand{\fakesection}[1]{%
% Increase section counter
  % Add section mark (header)
  \addcontentsline{toc}{section}{\protect\numberline{\thesection}#1}% Add section to ToC
  % Add more content here, if needed.
}
\renewcommand{\headrulewidth}{0.5pt}
\renewcommand{\footrulewidth}{0.5pt}
\newcommand{\horrule}[1]{\rule{\linewidth}{#1}}
\pagestyle{fancy}
\newcommand\invisiblesection[1]{%
  \refstepcounter{section}%
  \addcontentsline{toc}{section}{\protect\numberline{}Feladatkitűzés}
  \sectionmark{1}}
\newcommand{\NEV}{Gergő Kelle}
\newcommand{\NEPTUN}{GBBNUL}
\newcommand{\TANTARGYKOD}{BMEGEMMNWFE}
\newcommand{\HFCIME}{Finite Element Analysis \\ II. Homework}
\newcommand{\HFFEJLECCIME}{Finite Element Analysis}
\title{\centering \includegraphics[width=10 cm , height=2.8cm]{bme_logo_nagy}
\bigskip

\normalfont \normalsize \textsc{\centering Budapest University of Technology and Economics
\\ Faculty of Mechanical Engineering} \\ [12pt] \horrule{0.5pt} \\[0.4cm] \huge \HFCIME{} \\ \horrule{2pt} \\[0.5cm]}
\lhead{\TANTARGYKOD, \HFFEJLECCIME{}}
\rfoot{\thepage\ / \pageref{LastPage}}
\cfoot{}
\rhead{\NEV{}, \NEPTUN{}}
\author{\NEV{}}
\lstdefinestyle{mystyle}{
    backgroundcolor=\color{backcolour},   
    commentstyle=\color{codegreen},
    keywordstyle=\color{magenta},
    numberstyle=\tiny\color{codegray},
    stringstyle=\color{codepurple},
    basicstyle=\ttfamily\footnotesize,
    breakatwhitespace=false,         
    breaklines=true,                 
    captionpos=b,                    
    keepspaces=true,                 
    numbers=left,                    
    numbersep=5pt,                  
    showspaces=false,                
    showstringspaces=false,
    showtabs=false,                  
    tabsize=2
}

\lstset{style=mystyle}

\begin{document}
\includepdf[pages=1]{Finite_Element_Analysis_HW_02_filled.pdf}
\pagebreak
\maketitle
\pagebreak

\pagebreak
\tableofcontents
\newpage
\section{Equation of motion on a multi-DOF system}
Apply the equation of motion of multi-DoF systems! Use four \textit{BEAM1D} elements to create the structural matrices for the finite element analysis! Apply the consistent mass matrix and consider the cases of conservative \circled{a.} and nonconservative \circled{b.} force!\\
\horrule{0.4pt}

\subsection{Parameters and discretization}

\begin{table}[htbp]
    \centering
    \caption{Parameters for Homework ID 121}
    \label{tab:parameters-121}
    \begin{tabular}{ccc}
        Parameter & Value & Dimension\\ \hline
        L & 0.5 &m \\
        d & 10 &mm \\
        D & 50 &mm \\
        h & 10 &mm \\
        $\rho_{beam}$ & 7730 &kg/m$^3$ \\
        $\rho_{disk}$ & 7410 &kg/m$^3$ \\
        $E_{beam}$ & 205& GPa
    \end{tabular}
\end{table}

Later calculations require the definition of some basic parameters which are the following:
\begin{gather}
m_{disk} = \dfrac{D^2 \pi}{4} \cdot \rho_{disk} ~ h = 0.1455 ~ \left[ \text{kg} \right], \\
A = \dfrac{d^2 \pi}{4} = 1.9635 \cdot 10^{-3} ~ \left[ \text{m} ^2 \right],\\
I_z = \dfrac{d^4 \pi}{64} = 4.9087 \cdot 10^{-10} ~ \left[ \text{m} ^4 \right], \\
\theta_{disk} = \frac{1}{4} \cdot m_{disk} \left( \frac{D}{2} \right)^2 + \frac{1}{12} \cdot m_{disk} ~ h^2 = 2.3946 \cdot 10^{-5} ~ \left[ \text{kg m} ^2 \right].
\end{gather}
Due to the circular shape of the beam's cross section remaining constant along the X-axis, the area and second moment of inertia remains consistent throughout the beam. Before further calculations discretization must be done where in regard to the task \textit{BEAM1D} models should be used. The information about discretization showed it Table \ref{tab:disc}.

\begin{table}[H]
\label{tab:disc}
\captionsetup{justification=centering,position=top} % set caption above table and centered
\caption{Discretization} % common caption for both tables
\begin{minipage}[t]{0.45\linewidth}
\centering
\captionsetup{justification=centering} % set subcaption centered
\begin{tabular}{ c c c }
 Node & x $\left[ \text{mm} \right]$ & y $\left[ \text{mm} \right]$ \\
 \hline
 1 & 0 & 0 \\
 2 & 125 & 0 \\
 3 & 250 & 0 \\
 4 & 375 & 0 \\
 5 & 500 & 0 \\
\end{tabular}
\subcaption{Node coordinates} % subcaption for first table
\label{tab:node-coordinates}
\end{minipage}
\quad
\begin{minipage}[t]{0.45\linewidth}
\centering
\captionsetup{justification=centering} % set subcaption centered
\begin{tabular}{ c c c }
 Element & Local node 1 & Local node 2 \\
 \hline
 1 & 1 & 2 \\
 2 & 2 & 3 \\
 3 & 3 & 4 \\
 4 & 4 & 5 \\
\end{tabular}
\vspace{0.24 cm}
\subcaption{Element connectivity} % subcaption for second table
\label{tab:element-connectivity}
\end{minipage}
\end{table}

\pagebreak
\subsection{Equation of motion}

\circled{a.} conservative
\begin{equation}
\textbf{M}\ddot{\textbf{u}} + \left(\textbf{K} + \textbf{K}_G\right) \textbf{u} = \textbf{0}
\end{equation}
\circled{b.} nonconservative
\begin{equation}
\textbf{M}\ddot{\textbf{u}} + \left(\textbf{K} + \textbf{K}_G + \textbf{K}_L \right) \textbf{u} = \textbf{0}
\end{equation}
\begin{equation*}
	\hspace{-8cm}
	\begin{aligned}
	\text{Where} \quad 
	\begin{array}{|l@{}}
	\begin{tabular}{ l c l }
		\textbf{M}\textsubscript{e} &: &\text{Mass matrix}\\
		\textbf{K}\textsubscript{e} &: &\text{Stiffness matrix} \\
		\textbf{K}\textsubscript{G} &: &\text{Geometric stiffness matrix} \\
		\textbf{K}\textsubscript{L} &: &\text{Structural load stiffness matrix} \\
		\textbf{u} &: &\text{Vector of nodal displacements}
	\end{tabular}
	\end{array}
	\end{aligned}
	\end{equation*}
Equations for these parameters showed in Table \ref{tab:important_equations}.
	
\begin{table}[h]
\centering
\caption{Main equations}
\label{tab:important_equations}
\renewcommand{\arraystretch}{1.2}
\begin{tabular}{lc}
\textbf{Parameter} & \textbf{Equation} \\
\hline
\textbf{Consistent mass matrix} &  \parbox{10cm}{\begin{align} \textbf{M}_e = \frac{Al\rho}{420} \begin{bmatrix}
156 & 22l & 54 & -13l \\
22l & 4l^2 & 13l & -3l^2 \\
54 & 13l & 156 & -22l \\
-13l & -3l^2 & -22l & 4l^2
\end{bmatrix}  \end{align}} \\

\textbf{Material stiffness matrix} & \parbox{10cm}{\begin{align} \textbf{K}_e = \frac{I_1El^3}{12} \begin{bmatrix}
12 & 6l & -12 & 6l \\
6l & 4l^2 & -6l & 2l^2 \\
-12 & -6l & 12 & -6l \\
6l & 2l^2 & -6l & 4l^2
\end{bmatrix}  \end{align}} \\

\textbf{Geometric stiffness matrix} & \parbox{10cm}{\begin{align}
\textbf{K}_{Ge} = \frac{N}{30l} \begin{bmatrix}
36 & 3l & -36 & 3l \\
3l & 4l^2 & -3l & -l^2 \\
-36 & -3l & 36 & -3l \\
3l & -l^2 & -3l & 4l^2
\end{bmatrix}
\end{align}}  \\

\textbf{Structural load stiffness matrix} & \parbox{10cm}{\begin{align}
\label{eq:K_L}
\textbf{K}_{L} = -\dfrac{\partial \textbf{F}_{nc}}{\partial \textbf{u}}
\end{align}}  \\

\end{tabular}
\end{table}
\noindent The stiffness and mass matrices of all four discretized elements are identical due to the beam's uniform cross-section and isotropic material properties. Thus the global matrices are obtained by summing the identical mass or stiffness matrices with regard to it's DoF.
This way we get 10 $\cross $10 matrices that has to be condensed based on the boundary conditions of $v_1 = 0$ and $\varphi = 0$ resulting the ellimination of the 1st and 2nd rows and columns of the structural matrices. Condensed matrices  will be 8 $\cross $ 8 matrices.\newline
Another thing to point out is the mass matrix for the whole structure, because we have to consider the disk. Thus before the structural mass condensation, we should substitute the beam's parameters into $M_{e}$ then calculate the global structural mass matrix of the beam $M_{beam}$ and add the disk's mass matrix $M_{disk}$ leaving us with the following global mass matrix:
\begin{equation}
\textbf{M} = \textbf{M}_{beam} + \textbf{M}_{disk}
\end{equation}
\begin{equation*}
	\hspace{-7.5 cm}
	\begin{aligned}
	\text{Where} \quad 
	\begin{array}{|l@{} c l}
M_{disk} &~=& \begin{bmatrix}
0 & 0 & 0 & \cdots & 0 \\
0 & 0 & \cdots & \cdots & 0 \\
0 & \vdots & \ddots & 0 & 0 \\
\vdots & \vdots & 0 & m_{disk} & 0 \\
0 & 0 & 0 & 0 & \theta_{disk} 
\end{bmatrix}
	\end{array}
	\end{aligned}
\end{equation*}
After we have the condensated matrices for the conservative force, we examine the divergence instability, and for that purpose the corresponding equation of motion is
\begin{equation}
\textbf{M}_c \ddot{\textbf{u}}_c + \left( \textbf{K}_c + \textbf{K}_{Gc}  \right) \textbf{u}_c = \textbf{0}
\end{equation}

For case \circled{b.}, the nonconservative part of the force vector depends on the \textbf{u} nodal displacement. The nonconservative part of the force is
\begin{equation}
F_{nc}^T = \begin{bmatrix}
0 &0 &0 &0 &0 &0 & -F \sin\varphi_5 & 0
\end{bmatrix}
\end{equation}
As we are dealing with small values of $\varphi$, we can simplify and linearize the expression by expanding it in a Taylor series:
\begin{equation*}
\sin{\varphi} = \varphi - \xcancel{\frac{\varphi^3}{3!} + ~ \dots~}
\end{equation*}
thus 
\begin{equation*}
\sin \varphi \approx \varphi
\end{equation*}
leaving us with a linearized nonconservative part of the force which is the following:
\begin{equation}
\label{eq:F_c}
F_{nc}^T = \begin{bmatrix}
0 &0 &0 &0 &0 &0 & -F \varphi_5 & 0
\end{bmatrix}
\end{equation}
From (\ref{eq:F_c}) equation (\ref{eq:K_L}) can be determined, therefore the nonconservative force case of the equation of motion can be calculated:
\begin{equation}
\textbf{M}_c \ddot{\textbf{u}}_c + \left( \textbf{K}_c + \textbf{K}_{Gc} + \textbf{K}_{Lc}  \right) \textbf{u}_c = \textbf{0}
\end{equation}
\newpage

\section{Characteristic equation}
Expand the characteristic equation of the system!\\
\horrule{0.4pt}
\medskip
To analyze both cases \circled{a.} and \circled{b.}, we need to start by expanding the characteristic equations of the system, and then create the root amplitude plots by considering the first four roots of the eigenequation. The test function for both cases can be expressed using 
\begin{gather}
\textbf{u}_c (t) = C \boldsymbol{\phi}_c \text{e} ^{pt}, \\
\ddot{\textbf{u}} (t) = p^2 C \boldsymbol{\phi}_c \text{e} ^{pt} .
\end{gather}



\begin{equation*}
	\hspace{-9.5 cm}
	\begin{aligned}
	\text{Where} \quad 
	\begin{array}{|l@{} c l}
C &: &\text{A constant}\\
\boldsymbol{\phi}_c &: &\text{The mode shape vector} \\
p &: &\text{The characteristic exponent}\\
	\end{array}
	\end{aligned}
\end{equation*}
	
\noindent Using the test function in the equation of motion for both cases and simplifying it, we obtain the following equation in regards of \circled{a.} case
 \begin{equation}
\left( p^2 \textbf{M}_c + \textbf{K}_c + \textbf{K}_{Gc}  \right) C \boldsymbol{\phi}_c \text{e} ^{pt} = \textbf{0}.
\end{equation}
By solving this equation, we can derive the nontrivial solution, which leads us to the eigenequation of
 \begin{equation}
\det \left( p^2 \textbf{M}_c + \textbf{K}_c + \textbf{K}_{Gc}  \right) = 0, 
\end{equation}
As for \circled{b.} case the corresponding equation of motion leads to the following eigenequation:
\begin{equation}
\left( p^2 \textbf{M}_c  + \textbf{K}_c + \textbf{K}_{Gc} + \textbf{K}_{Lc}  \right) C \boldsymbol{\phi}_c \text{e} ^{pt} = \textbf{0}.
\end{equation}
\begin{equation}
\det \left( p^2 \textbf{M}_c  + \textbf{K}_c + \textbf{K}_{Gc} + \textbf{K}_{Lc}  \right)  = 0.
\end{equation}
In order to calculate the $\textbf{K}_{Gc}$ and $\textbf{K}_{Lc}$, we need to solve the eigenequations for different values of the $F$ force at the right end. To simplify the calculations, we can use dimensionless parameters $\nu$ and $\lambda$ instead of the original $p$ and $F$ parameters. The characteristic exponent can be expressed as a function of $\nu$ and $ \lambda$.
\begin{equation}
\label{eq:p}
p = \sqrt{\nu \dfrac{Iz ~ E_{beam}}{A ~ L^4 ~ \rho_{beam}}}
\end{equation}
and the force can be written as
\begin{equation}
\label{eq:F}
F = \lambda \dfrac{I_z ~ E_{beam}}{L^2}
\end{equation}
\begin{equation*}
	\hspace{-9.5 cm}
	\begin{aligned}
	\text{Where} \quad 
	\begin{array}{|l@{} c l}
\nu &: &\text{The frequency parameter}\\
\lambda &: &\text{The load control parameter} \\
	\end{array}
	\end{aligned}
\end{equation*}
As in a more representative form for later visualization (\ref{eq:p}) and (\ref{eq:F}) can be written in the following form:
\begin{gather}
\sqrt{\nu} = \sqrt{\dfrac{\rho_{beam} ~ A ~ L^4 ~ p^2}{I_z ~ E_{beam}}} \\
\lambda = \dfrac{F ~ L^2}{I_z ~ E_{beam}}
\end{gather}
\newpage
\section{Root amplitude plots}
Create the root amplitude plots considering the first four roots of eigenequation of the
system for cases \circled{a.} and \circled{b.} !\\
\horrule{0.4pt}
\medskip
\begin{figure}[htbp]
	\centering
	\begin{minipage}[b]{0.5\textwidth}
	\centering
	\includegraphics[width=\linewidth]{rajz-1.pdf}
	\subcaption{Root amplitude plot for divergence instability}
	\label{fig:root_amp}
	\end{minipage}\hfill
	\begin{minipage}[b]{0.45\textwidth}
	\centering
	\includegraphics[width=\linewidth]{rajz-2_v3.pdf}
	\subcaption{Root amplitude plot for flutter instability}
	\label{fig:root_amp_flutter}
	\end{minipage}
	\caption{Root amplitude plots for divergence and flutter instability (step size: $\delta \lambda = 0.1$)}
	\label{fig:root_amp_comparison}
\end{figure}
\section{Task 4 \& 5}
Determine the critical loads (or load control parameters) for divergence \circled{a.} and flutter \circled{b.} instability of the system! Plot the time response of the system!\newline
Calculate the mode shapes at flutter instability using the first two critical loads (or load control parameters) when \circled{S} $\rightarrow$ \circled{U} and display the shapes! Create the phase plane portraits!\\
\horrule{0.4pt}
\medskip
This task requires to determine the load control parameters that lead to divergence and flutter instability of the system, and to create time response plots for each scenario. To calculate the critical load control parameters for \circled{a.} divergence instability, we need to solve an eigenequation with a frequency parameter of zero, which (as Figure \ref{fig:root_amp_comparison} shows) gives us four solution that Table \ref{tab:divergence-lambda} contains.
\begin{table}[h]
\centering
\caption{Critical Load Control Parameters for Divergence Instability}
\begin{tabular}{cc}
\hline
Critical Load Control Parameter & Value \\
\hline
$\lambda_{1}$ & 2.4675 \\

$\lambda_{2}$ & 22.2621 \\

$\lambda_{3}$ & 62.7526 \\

$\lambda_{4}$ & 127.468 \\

\end{tabular}

\label{tab:divergence-lambda}
\end{table}

\noindent As for flutter instability, we need to use numerical analysis to estimate the critical load control parameters. When the system experiences flutter instability, two roots merge and form a complex conjugate pair. The estimated critical load control parameters for flutter instability \circled{b.} are 
$$\lambda_{cr1} \approx 16.1~,$$ 
$$\lambda_{cr2} \approx 115 $$ 
which are shown in Figure \ref{fig:root_amp_flutter}. The corresponding force value can be calculated easily with Eq. (\ref{eq:F}). Finally, we need to create time response plots for both types of instability. At these points the system goes from a stable stance to an unstable stance \circled{S} $\rightarrow$ \circled{U} thus i plotted the results above those critical $\lambda$ values. The resulting phase plane portraits are showed in Figure \ref{fig:div1_2}, \ref{fig:div1_2_2}, \ref{fig:flutter1_2} and \ref{fig:phase}.

\begin{figure}[H]
	\centering
	\begin{minipage}[b]{0.47\textwidth}
	\centering
	\includegraphics[width=\linewidth]{time_div_l2_5_VV.pdf}
	\subcaption{Around $\lambda_1$}
	\label{fig:root_amp}
	\end{minipage}\hfill
	\begin{minipage}[b]{0.47\textwidth}
	\centering
	\includegraphics[width=\linewidth]{time_div_l23_VV.pdf}
	\subcaption{Around $\lambda_2$}
	\label{fig:root_amp_div}
	\end{minipage}
	\caption{Time response of the system at the first and second divergence instability state}
	\label{fig:div1_2}
\end{figure}

\begin{figure}[H]
	\centering
	\begin{minipage}[b]{0.47\textwidth}
	\centering
	\includegraphics[width=\linewidth]{phase_div_l2_5.pdf}
	\subcaption{At $\lambda = 2.5$}
	\label{fig:root_amp}
	\end{minipage}\hfill
	\begin{minipage}[b]{0.47\textwidth}
	\centering
	\includegraphics[width=\linewidth]{phase_div_l23.pdf}
	\subcaption{At $\lambda = 23$}
	\label{fig:root_amp_div}
	\end{minipage}
	\caption{Phase plane portrait for divergence instability above the first two critical load parameter}
	\label{fig:div1_2_2}
\end{figure}

\begin{figure}[htbp]
	\centering
	\begin{minipage}[b]{0.47\textwidth}
	\centering
	\includegraphics[width=\linewidth]{time_l18_VV.pdf}
	\subcaption{Around $\lambda_{cr1}$}
	\label{fig:root_amp}
	\end{minipage}\hfill
	\begin{minipage}[b]{0.47\textwidth}
	\centering
	\includegraphics[width=\linewidth]{time_l118_VV.pdf}
	\subcaption{Around $\lambda_{cr2}$}
	\label{fig:root_amp_flutter}
	\end{minipage}
	\caption{Time response of the system around the first and second flutter instability state}
	\label{fig:flutter1_2}
\end{figure}


\begin{figure}[H]
	\centering
	\begin{minipage}[b]{0.47\textwidth}
	\centering
	\includegraphics[width=\linewidth]{phase_l18.png}
	\subcaption{$\lambda = 18$}
	\label{fig:phase1}
	\end{minipage}\hfill
	\begin{minipage}[b]{0.47\textwidth}
	\centering
	\includegraphics[width=\linewidth]{phase_l118.png}
	\subcaption{$\lambda = 118$}
	\label{fig:phase2}
	\end{minipage}
	\caption{Phase plain portraits above the first and second flutter instability state}
	\label{fig:phase}
\end{figure}
\newpage
\noindent The mode shapes at flutter instability should be investigated around the load control parameter of flutter type of loss of stability, namely at $\lambda_{cr1}$ and $\lambda_{cr2}$ showed in Figure \ref{fig:mode}. From the mode shapes we can see that modes are merging together, at Figure \ref{fig:mode1} 2 modes merged, while Figure \ref{fig:mode2} shows that 2 pair of modes merged.


\begin{figure}[htbp]
	\centering
	\begin{minipage}[b]{0.47\textwidth}
	\centering
	\includegraphics[width=\linewidth]{mode_l18.pdf}
	\subcaption{$\lambda = 18$}
	\label{fig:mode1}
	\end{minipage}\hfill
	\begin{minipage}[b]{0.47\textwidth}
	\centering
	\includegraphics[width=\linewidth]{mode_l118.pdf}
	\subcaption{$\lambda = 118$}
	\label{fig:mode2}
	\end{minipage}
	\caption{Phase plain portraits above the first and second flutter instability state}
	\label{fig:mode}
\end{figure}

\noindent For comparison calculating with $\lambda = 0$ gives us the mode shapes with no prestress effect where modes does not merge together, this state is presented in Figure \ref{fig:mode_prestress}.

\begin{figure}[htbp]
\centering
\includegraphics[width=0.55 \linewidth]{mode_l0.pdf}
\caption{Mode shapes without prestress effect}
\label{fig:mode_prestress}
\end{figure}

\end{document}
