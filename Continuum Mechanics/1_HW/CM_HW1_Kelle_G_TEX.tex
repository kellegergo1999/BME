\documentclass[12pt]{article}
\usepackage{blindtext}
\usepackage[top=25mm, bottom=25mm, left=20mm, right=20mm]{geometry}
\usepackage{t1enc}
\usepackage{fancyhdr}
\usepackage[english]{babel}
\usepackage{lastpage}
\usepackage{amsmath}
\usepackage{amssymb,physics}
\usepackage{amsthm}
\usepackage{empheq}
\usepackage{graphicx}
\usepackage{wrapfig}
\usepackage{tikz}
\usepackage{bm}
\usepackage{gensymb}
\usepackage{pdfpages}
\usepackage{hyperref}
\usepackage{multirow}
\usepackage{mdframed}
\usepackage[most]{tcolorbox}
\usepackage{pgfplots}
\usepackage{float}
\usepackage{caption}
\usepackage{subcaption}
\pgfplotsset{compat = newest}
\usepackage{listings}
\usepackage[makeroom]{cancel}
\usepackage{xcolor}

\definecolor{codegreen}{rgb}{0,0.6,0}
\definecolor{codegray}{rgb}{0.5,0.5,0.5}
\definecolor{codepurple}{rgb}{0.58,0,0.82}
\definecolor{backcolour}{rgb}{0.9059,0.902,0.9137}
\hypersetup{
    colorlinks=true,
    linkcolor=black,
    filecolor=magenta,      
    urlcolor=cyan,
}
\newtcolorbox{mygraybox}[1]{
  enhanced,
  colback = gray!10,
  colbacklower = black!80,
  colframe = black,
  rounded corners,
  boxrule = 0pt,
  leftrule = 2pt, % Adjust this for a thicker black frame
  toprule = 0pt, % Adjust this for a thicker black frame
  rightrule = 0pt, % Adjust this for a thicker black frame
  bottomrule = 0pt, % Adjust this for a thicker black frame
  left = 10pt,
  right = 10pt,
  top = 2pt,
  bottom = 2pt,
  arc = 1mm, % Adjust this for the roundness
  before upper = #1
}


\urlstyle{same}
\usepackage{graphicx}
\graphicspath{ {./images/} }
\usepackage{pict2e}
\usepackage{blindtext}
\usepackage{tikz}
\newcommand*\circledB{%
  \tikz[baseline=(char.base)]{%
    \node[shape=circle,draw,inner sep=1pt] (char) {b};%
  }%
}
% Table float box with bottom caption, box width adjusted to content


\newcommand*\circled[1]{%
  \tikz[baseline=(char.base)]{%
    \node[shape=circle,draw,inner sep=1pt] (char) {#1};%
  }%
}


\DeclareRobustCommand{\slashcirc}{{\mathpalette\doslashcirc\relax}}

\usepackage{graphicx,array}
\newcolumntype{C}[1]{>{\centering\let\newline\\\arraybackslash\hspace{130pt}}m{#1}}
\newcolumntype{L}[1]{>{\raggedright\let\newline\\\arraybackslash\hspace{130pt}}m{#1}}

\makeatother

\usepackage{amsmath}
\newcommand\aug{\fboxsep=-\fboxrule\!\!\!\fbox{\strut}\!\!\!}
\usepackage{wasysym}
\usepackage{tikz}
\usepackage{xcolor}
%\def\checkmark{\tikz\fill[scale=0.4](0,.35) -- (.25,0) -- (1,.7) -- (.25,.15) -- cycle;} 
\newcommand*\widefbox[1]{\fbox{\hspace{2em}#1\hspace{2em}}}
\newcommand{\fakesection}[1]{%
% Increase section counter
  % Add section mark (header)
  \addcontentsline{toc}{section}{\protect\numberline{\thesection}#1}% Add section to ToC
  % Add more content here, if needed.
}
\renewcommand{\headrulewidth}{0.5pt}
\renewcommand{\footrulewidth}{0.5pt}
\newcommand{\horrule}[1]{\rule{\linewidth}{#1}}
\pagestyle{fancy}
\newcommand\invisiblesection[1]{%
  \refstepcounter{section}%
  \addcontentsline{toc}{section}{\protect\numberline{}Feladatkitűzés}
  \sectionmark{1}}
\newcommand{\NEV}{Gergő Kelle}
\newcommand{\NEPTUN}{GBBNUL}
\newcommand{\TANTARGYKOD}{BMEGEMMNWCM}
\newcommand{\HFCIME}{Continuum Mechanics \\ I. Homework}
\newcommand{\HFFEJLECCIME}{Continuum Mechanics}
\title{\centering \includegraphics[width=10 cm , height=2.8cm]{figures/bme_logo_nagy}
\bigskip

\normalfont \normalsize \textsc{\centering Budapest University of Technology and Economics
\\ Faculty of Mechanical Engineering} \\ [12pt] \horrule{0.5pt} \\[0.4cm] \huge \HFCIME{} \\ \horrule{2pt} \\[0.5cm]}
\lhead{\TANTARGYKOD, \HFFEJLECCIME{}}
\rfoot{\thepage\ / \pageref{LastPage}}
\cfoot{}
\rhead{\NEV{}, \NEPTUN{}}
\author{\NEV{}}
\lstdefinestyle{mystyle}{
    backgroundcolor=\color{backcolour},   
    commentstyle=\color{codegreen},
    keywordstyle=\color{magenta},
    numberstyle=\tiny\color{codegray},
    stringstyle=\color{codepurple},
    basicstyle=\ttfamily\footnotesize,
    breakatwhitespace=false,         
    breaklines=true,                 
    captionpos=b,                    
    keepspaces=true,                 
    numbers=left,                    
    numbersep=5pt,                  
    showspaces=false,                
    showstringspaces=false,
    showtabs=false,                  
    tabsize=2
}

\lstset{style=mystyle}

\begin{document}
\includepdf[pages=1]{figures/contmech_hw1_sajat.pdf}
\pagebreak
\maketitle
\pagebreak

\pagebreak
\tableofcontents
\newpage
\section{Position of the reference and the spatial configurations}
Key informations: It is a 4-node linear plane strain quad element. 1-point Gaussian quadrature is used in the element formulation. The material particle located at the Gauss integration point is labelled with G.\\
\horrule{0.4pt}

\noindent As the Homework instruction page suggest I downloaded the figure about the problem with my NEPTUN code, which can be seen at fig. \ref{fig:your-figure-label}. 

  \begin{figure}[ht!]
    \begin{center}
    \includegraphics[width=0.55\textwidth]{GBBNUL-HW1.pdf}
    \captionof{figure}{Your Figure Caption}
    \label{fig:your-figure-label}
    \end{center}
  \end{figure}

\noindent Since the node points and the corresponding vectors will be neccesarry I've listed them in Tab. \ref{table:pos}.

\begin{table}[ht!]
  \begin{center}
  \captionof{table}{Table of position vectors}
  \label{table:pos}
  \begin{tabular}{ccc}
    Node & Reference & Current \\[0.1 cm] \hline
    \textbf{1} & $\textbf{X}_{0,1}^T = \begin{bmatrix} 0 & 0 & 0 \end{bmatrix}$ & $\textbf{x}_{0,1}^T = \begin{bmatrix} 9 & -1 & 0 \end{bmatrix}$ \\[0.15cm]
    \textbf{2} & $\textbf{X}_{0,2}^T = \begin{bmatrix} 4 & 1 & 0 \end{bmatrix}$ & $\textbf{x}_{0,2}^T = \begin{bmatrix} 8 & 4 & 0 \end{bmatrix}$ \\[0.15cm]
    \textbf{3} & $\textbf{X}_{0,3}^T = \begin{bmatrix} 5 & 4 & 0 \end{bmatrix}$ & $\textbf{x}_{0,3}^T = \begin{bmatrix} 6 & 2 & 0 \end{bmatrix}$ \\[0.15cm]
    \textbf{4} & $\textbf{X}_{0,4}^T = \begin{bmatrix} 1 & 2 & 0 \end{bmatrix}$ & $\textbf{x}_{0,4}^T = \begin{bmatrix} 7 & -2 & 0 \end{bmatrix}$ \\
  \end{tabular}
\end{center}
\end{table}

\noindent The undeformed (reference) and the deformed (current) position vectors are compiled according to the Tab. \ref{table:pos} in Eq. \ref{eq:1} and \ref{eq:2}.


\begin{minipage}{0.4\textwidth}
  \label{eq:1}  
  \begin{equation}
      \textbf{X} = \begin{bmatrix}
        0 & 0 & 0 \\
        4 & 1 & 0 \\
        5 & 4 & 0 \\
        1 & 2 & 0
      \end{bmatrix} \! \text{mm}
    \end{equation}
\end{minipage}%
\begin{minipage}{0.4\textwidth}
  \label{eq:2}  
  \begin{equation}
      \textbf{x} = \begin{bmatrix}
        9 & -1 & 0 \\
        8 & 4 & 0 \\
        6 & 2 & 0 \\
        7 & -2 & 0
      \end{bmatrix} \! \text{mm}
    \end{equation}
\end{minipage}%

Before any further calculations, the position of the Gauss point (\textbf{G}) needs to be determined. Since \textbf{G} is positioned at the quad element center the average of every corresponding node coordinate value will give us the solution.

\begin{minipage}{0.35\textwidth}
      \label{eq:gauss_point}
      
      \begin{gather}
        \begin{align}
          G_x &= \frac{1}{k} \sum_{i=1}^{k} \textbf{X}_{i,1} \\
          G_y &= \frac{1}{k} \sum_{i=1}^{k} \textbf{X}_{i,2} \\
          G_z &= \frac{1}{k} \sum_{i=1}^{k} \textbf{X}_{i,3} \\[0.1cm]
          \textbf{X}_G^T &= \begin{bmatrix}
            2.5 & 1.75 & 0
          \end{bmatrix} \\[0.1cm]
          \textbf{x}_G^T &= \begin{bmatrix}
            7.5 & 0.75 & 0
          \end{bmatrix}
        \end{align} 
      \end{gather}
\end{minipage}%
\begin{minipage}{0.1\textwidth}
\space \!
\end{minipage}
\begin{minipage}{0.5\textwidth}
  \hspace{0.55 cm}
  \begin{lstlisting}[language=Python]
def calculate_gauss_point(nodes):
  num_nodes = len(nodes)
  gx, gy, gz = 0, 0, 0
  for x, y, z in nodes:
      gx += x
      gy += y
      gz += z
  gx /= num_nodes
  gy /= num_nodes
  gz /= num_nodes
  return gx, gy, gz

undeformed_gauss_point = calculate_gauss_point(undeformed_nodes)
deformed_gauss_point = calculate_gauss_point(deformed_nodes)
\end{lstlisting}
\end{minipage}

\begin{mygraybox}{\textit{i: For later calculations the whole code is available as \hyperref[page1]{appendix}.}}
\end{mygraybox}

\section{Displacement vector}
The displacement vector of \textbf{G} can be determined as the difference of reference and current position
\begin{equation}
  \textbf{U}_G^T =  \textbf{x}_G^T - \textbf{X}_G^T = \begin{bmatrix}
    5.0 & -1.0 & 0
  \end{bmatrix} \!.
\end{equation} 
Displacement can be determined for every node as
\begin{equation}
  \textbf{U}_{nodes} =  \textbf{x} - \textbf{X} = \begin{bmatrix}
    9 & -1 & 0 \\
    4 & 3 & 0\\
    1 & -2 & 0\\ 
    6 & -4 & 0
  \end{bmatrix}
\end{equation}.
\section{Deformation gradient}
Deformation gradiend most commonly calculated with one of the following methods:
\begin{equation}
  \textbf{F}(\textbf{X}) = \text{Grad} \bm{\chi} (\textbf{X},t) = \dfrac{\partial \bm{\chi} (\textbf{X},t)}{\partial \textbf{X}} \quad \text{or} \quad \textbf{F}(\textbf{X}) = \text{Grad} \textbf{U}(\textbf{X},t) + \textbf{I}
\end{equation}

\begin{equation*}
	\hspace{-5 cm}
	\begin{aligned}
	\text{Where}
	 \quad
	  \begin{array}{|l@{} c l}
		  \text{Deformation gradient}  & :  & \textbf{F}(\textbf{X},t) \\
		  \text{Motion}  & :  & \bm{\chi}(\textbf{X},t) \\
      \text{Material displacement field } & : & \textbf{U}(\textbf{X},t) \\
      \text{Material displacement gradient } & : & \textbf{K}(\textbf{X},t) = \text{Grad} \textbf{U}(\textbf{X},t)
	  \end{array}
  \end{aligned}
\end{equation*}

The determination of the motion function from the provided data for the computation of the deformation gradient remains unattainable. Therefore calculating the material displacement field is the better solution.
Utilizing the shape functions as Equation \ref{eq:shapeF} shows, the displacement field can be computed using the nodal displacements.
\begin{gather}
  \label{eq:shapeF}
  N_1(\xi, \eta) = \frac{1}{4} \left(1 - \xi\right) \left(1 - \eta\right) \\
  N_2(\xi, \eta) = \frac{1}{4} \left(1 + \xi\right) \left(1 - \eta\right) \\
  N_3(\xi, \eta) = \frac{1}{4} \left(1 + \xi\right) \left(1 + \eta\right) \\
  N_4(\xi, \eta) = \frac{1}{4} \left(1 - \xi\right) \left(1 + \eta\right)
\end{gather}

\begin{mygraybox}
  {\textit{i: To elucidate the notational conventions, as illustrated in Figure \ref{fig:your-figure-label}, we denote $X_1$ as the nonzero coordinate of the vector $\textbf{E}_1$, and $X_2$ as the nonzero coordinate of $\textbf{E}_2$ (analogously, $X_3$).}}
\end{mygraybox}

\begin{gather}
  X_1(\xi, \eta) = \sum_{i = 1}^{k} \textbf{X}_{i1} \cdot N_i(\xi, \eta) \\
  X_2(\xi, \eta) = \sum_{i = 1}^{k} \textbf{X}_{i2} \cdot N_i(\xi, \eta)
\end{gather}

To obtain the displacement field components, we must transform the $\xi$ and $\eta$ coordinates with respect to $X_1$ and $X_2$. 
This involves inverting the previously mentioned expressions. The inversion yields two separate solutions, 
and we must choose the suitable one. 
The resultant solutions are as follows:


\begin{table}[ht!]
  \begin{center}
  \captionof{table}{Inverse function solutions}
  \label{table:sol_val}
  \renewcommand{\arraystretch}{1.5} % Increase the vertical spacing
  \begin{tabular}{lll}
    & $\xi(X_1, X_2)$ & $\eta(X_1, X_2)$ \\ \hline
    solution 1 & $\frac{1}{4} \left(-11 + X_1 - \sqrt{49 + 18 X_1 + X_1^2 - 16 X_2}\right)$ & $6 + X_1 + \sqrt{49 + 18 X_1 + X_1^2 - 16 X_2}$ \\
    solution 2 & $\frac{1}{4} \left(-11 + X_1 + \sqrt{49 + 18 X_1 + X_1^2 - 16 X_2}\right)$ & $6 + X_1 - \sqrt{49 + 18 X_1 + X_1^2 - 16 X_2}$
  \end{tabular}
  \end{center}
\end{table}

\newpage
The selected solution should satisfy the condition where $\xi$ = 0 and $\eta$ = 0 for the Gauss point's coordinates ($X_G$)
 in the local coordinate system. 
 \begin{table}[ht!]
  \begin{center}
    \captionof{table}{Solutions given value at $\xi$ = 0 and $\eta$ = 0}
    \label{table:sol_val}
  \begin{tabular}{cll}
             & $\xi(X_1, X_2)$                                                            & $\eta(X_1, X_2)$                               \\ \hline
  solution 1 & $-4.25$ & $17$ \\
  solution 2 & $0$ & $0 $
  \end{tabular}
  \end{center}
\end{table}

Thus the second

\begin{gather}
  \xi(X_1, X_2) = \frac{1}{4} \left(-11 + X_1 + \sqrt{49 + 18 X_1 + X_1^2 - 16 X_2}\right)  \text{,}\\
  \eta(X_1, X_2) = 6 + X_1 - \sqrt{49 + 18 X_1 + X_1^2 - 16 X_2}
\end{gather}
solutions shall be used. Using these functions the material displacement field can be determined using the displacement of the nodes ($\textbf{U}_{nodes}$)

\begin{equation}
  \hspace{-1.2 cm}
  \textbf{U}(X_1, X_2, X_3) = \begin{bmatrix}
  \sum_{i=1}^{4} U_{nodes,i1} \cdot N_i(X_1, X_2)\\
  \sum_{i=1}^{4} U_{nodes,i2} \cdot N_i(X_1, X_2)\\
  0
  \end{bmatrix} = \begin{bmatrix}
    \frac{1}{8} \left(23 - 17X_1 + 7\sqrt{49 + 18X_1 + X_1^2 - 16X_2} \right) \\
    \frac{1}{4} \left(-11 + 5X_1 + \sqrt{49 + 18X_1 + X_1^2 - 16X_2} - 8X_2\right)\\
    0
 \end{bmatrix}
\end{equation}
\begin{equation}
  \textbf{U}(\textbf{X}_G)^T = \begin{bmatrix}
    5 & -1 & 0
  \end{bmatrix} = \textbf{x}_G^T - \textbf{X}_G^T \quad \checkmark
\end{equation}

From the material displacement field the deformation gradient can be calculated as
\begin{equation}
  \textbf{F}(\textbf{X}_G) = \text{Grad} \textbf{U}(\textbf{X}_G) + \textbf{I} = \begin{bmatrix}
    0.0588 & -0.8235 & 0 \\
    1.588	& -1.235	&0\\
    0& 0 & 1
 \end{bmatrix}.
\end{equation}

\section{Green-Lagrange and Euler-Almansi strain tensor}
In order to calculate Green-Lagrange strain tensor I've calculated Cauchy-Green deformation tensor beforehand. Since our task is to calculate every parameter for the Gauss point thus the numerical solutions are represented the solution at \textbf{G}.

\begin{gather}
  \textbf{C}(\textbf{X}) = \textbf{F}^T \textbf{F} \\
  \textbf{E}(\textbf{X}) = \frac{1}{2} \left( \textbf{C} - \textbf{I} \right) = \frac{1}{2} \left( \textbf{F}^T \textbf{F} - \textbf{I} \right)
\end{gather}
\begin{equation*}
	\hspace{-7 cm}
	\begin{aligned}
	\text{Where}
	 \quad
	  \begin{array}{|l@{} c l}
		  \text{Cauchy-Green deformation tensor}  & :  & \textbf{C}(\textbf{X})\\
		  \text{Green-Lagrange strain tensor}  & :  & \textbf{E}(\textbf{X}) \\
      \text{Second-order identity tensor}  & :  & \textbf{I}
	  \end{array}
  \end{aligned}
\end{equation*}
The resulting solutions are
\begin{gather}
  \label{Eq:23}
  \textbf{C}(\textbf{X}_G) =\begin{bmatrix}
    2.5252 & -2.0096 & 0 \\
    -2.0096 & 2.2034 & 0 \\
    0 & 0 & 1
  \end{bmatrix},\\
  \textbf{E}(\textbf{X}_G) = \begin{bmatrix}
    0.7626 & -1.0048 & 0 \\
    -1.0048 & 0.6017 & 0 \\
    0 & 0 & 0
  \end{bmatrix}.
\end{gather}

\begin{gather}
  \textbf{c} = \textbf{F}^{-T} ~ \textbf{F}^{-1} \\
  \textbf{e}(\textbf{X}) = \frac{1}{2} \left( \textbf{I} - \textbf{c} \right) = \frac{1}{2} \left( \textbf{I} - \textbf{F} ~ \textbf{F}^T \right)
\end{gather}
\begin{equation*}
	\hspace{-8.6 cm}
	\begin{aligned}
	\text{Where}
	 \quad
	  \begin{array}{|l@{} c l}
		  \text{Cauchy deformation tensor}  & :  & \textbf{c}(\textbf{X})\\
		  \text{Euler-Almansi strain tensor}  & :  & \textbf{e}(\textbf{X})
	  \end{array}
  \end{aligned}
\end{equation*}

The resulting solutions are
\begin{gather}
  \textbf{c}(\textbf{X}_G) =\begin{bmatrix}
    2.6530 & -0.72795 & 0 \\
    -0.72795 & 0.4468 & 0 \\
    0 & 0 & 1
  \end{bmatrix}, \\
  \textbf{e}(\textbf{X}_G) = \begin{bmatrix}
    -0.8265 & 0.3640 & 0 \\
    0.3640 & 0.2766 & 0 \\
    0 & 0 & 0
  \end{bmatrix}.
\end{gather}

\section{Volume element dv/dV ratio and volume strain}
Volume ratio and volume strain are both based on change of a volume element size between the reference and current configuration. Those parameters can be expressed as 

\begin{gather}
  J = \frac{\text{d}v}{\text{d}V} = \det \textbf{F}(\textbf{X}_G) =  1.2351\\
  \varepsilon_V = \frac{\text{d}v - \text{d}V}{\text{d}V} = J - 1 = 0.2351 \! .
\end{gather}
 
\begin{equation*}
	\hspace{-8 cm}
	\begin{aligned}
	\text{Where}
	 \quad
	  \begin{array}{|l@{} c l}
		  \text{Volume ratio}  & :  & J\\
      \text{Inf. small volume element (reference)}  & :  & \text{d}V\\
      \text{Inf. small volume element (current)}  & :  & \text{d}v\\
		  \text{Volume strain}  & :  & \varepsilon_V
	  \end{array}
  \end{aligned}
\end{equation*}

\section{Isochoric and volumetric parts of the deformation gradient}
The isochoric (volume-preserving) deformation stands for a deformation where the reference and current configuration element does not change in volume ($J=1$). 
Thus the deformation gradient (\textbf{F}) can be multiplicatively decomposed into an isochoric and volumetric
parts as 
\begin{equation}
  \textbf{F} = \textbf{F}_{iso} ~ \textbf{F}_{vol} = \textbf{F}_{vol} ~ \textbf{F}_{iso}
\end{equation} 
TheiIsochoric and volumetric parts can be calculated as
\begin{gather}
  \textbf{F}_{iso} (\textbf{X}_G) = J^{-\frac{1}{3}} \textbf{F}(\textbf{X}_G) = \begin{bmatrix}
    0.0548 & -0.7675 & 0 \\
    1.4800 & -1.1511 & 0 \\
    0 & 0 & 0.9320
  \end{bmatrix} \\
  \textbf{F}_{vol} (\textbf{X}_G) = J^{\frac{1}{3}} \textbf{I} = \begin{bmatrix}
    1.0729 & 0 & 0 \\
    0 & 1.0729 & 0 \\
    0 & 0 & 1.0729
  \end{bmatrix} \\
\end{gather}

To make sure the calculations were right we can check the following equilibrium

\begin{equation}
  \textbf{F}_{iso} ~ \textbf{F}_{vol} - \textbf{F} = \textbf{0} \! .
\end{equation}

Since our result satisfies the equilibrium above, the result is acceptable.
\section{Principal stretches}
Principal stretches can be calculated from the Cauchy-Green deformation tensor determined ar Eq. \ref{Eq:23}. The resulting eigenvalues can be found in Table \ref{table:eig}.

\begin{table}[ht!]
  \begin{center}
    \captionof{table}{Table of position vectors}
    \label{table:eig}
  \begin{tabular}{cl}
  Notation                   & Eigenvalue \\ \hline
  $\lambda_1 = \sqrt{\mu_1}$ & 2.0929         \\
  $\lambda_2 = \sqrt{\mu_2}$ & 1         \\
  $\lambda_3 = \sqrt{\mu_3}$ & 0.5901         
  \end{tabular}
  \end{center}
\end{table}

\section{Engineering strain of a line element}
To determine the engineering strain on the selected line element we have to calculate the length of the elemnt at the reference and the current state. The comparison of theese two length value will give us the engineering strain due to its deformation.
\begin{equation}
  \varepsilon_{e} = \dfrac{\left| \left| \textbf{x}_{0,2} - \textbf{x}_{0,4} \right| \right|}{\left| \left| \textbf{X}_{0,2} - \textbf{X}_{0,4} \right| \right|} -1 = 0.923538
\end{equation}

\section{Surface element da/dA ratio}
The surface element $\frac{\text{d}a}{\text{d}A}$ ratio can be determined similarly as the volume ratio.
In order to determine the surface element ration we need a vector thats perpendicular to it. Since the body is spread over two dimensions, the unit vector perpendicular to it is the unit vector of $\textbf{E}_3$ or $\textbf{e}_3$ direction.

\begin{gather}
  \textbf{N}_1^T = \textbf{E}_1^T = \begin{bmatrix}
    1 & 0 & 0
  \end{bmatrix}\\
  \textbf{N}_2^T = \textbf{E}_2^T = \begin{bmatrix}
    0 & 1 & 0
  \end{bmatrix}\\
  \textbf{N}_3^T = \textbf{E}_3^T = \begin{bmatrix}
    0 & 0 & 1
  \end{bmatrix}
\end{gather}

\begin{equation}
  \frac{\text{d}a}{\text{d}A} = \det \textbf{F} \sqrt{\textbf{N}_3~ \textbf{F}^{-1}~ \textbf{F}^{-T} ~\textbf{N}_3} = 1.2351
\end{equation}
The surface element ratio between the reference and the current configuration at \textbf{G} point is 1.2351.

\section{Angle of shear material line elements}
This task is about to determine the angle of shear in case of the material line elements parallel to the unit vectors of $\textbf{E}_1$ (or $\textbf{N}_1$) and $\textbf{E}_2$ (or $\textbf{N}_2$).
To be able to calculate the angle, the stretch ratios associated with these orientations needs to be determined with the following equations
\begin{gather}
  \lambda_{N_1} = \sqrt{\textbf{N}_1 ~ \textbf{C} ~\textbf{N}_1} = 1.5891\\
  \lambda_{N_2} = \sqrt{\textbf{N}_2 ~ \textbf{C} ~\textbf{N}_2} = 1.4844
\end{gather}

With the help of the stretch ratios the angle of shear can be determined as
\begin{equation}
  \gamma = \dfrac{\pi}{2} - \arccos \left( \dfrac{\textbf{N}_1 ~ \textbf{C} ~ \textbf{N}_2}{\lambda_{N_1} \lambda_{N_2}} \right) = -1.0197 \text{ rad} \quad \rightarrow \quad \gamma = -58.4259 ^\circ
\end{equation}

\section{Rigid body rotation}
The rigid body rotation tensor can be calculated with the following equations
\begin{gather}
  \mathbf{U} = \sqrt{\mathbf{C}}, \\
  \mathbf{R} = \mathbf{F} ~ \mathbf{U}^{-1} = \begin{bmatrix}
    -0.4384 & -0.8988 & 0 \\
    0.8988 & -0.4384 & 0 \\
    0 & 0 & 1
  \end{bmatrix}.
\end{gather}
To get the actual rotation angle and the deriction we can use the rotation tensor first scalar invariant value the following way:
\begin{gather}
  \theta = \arccos\left(\frac{\text{tr}(\mathbf{R}) - 1}{2}\right) = 2.0246 \text{ rad} \quad \rightarrow \quad \theta = 116^\circ \\
  \mathbf{N} = \frac{\mathbf{R} - \mathbf{R}^\top}{2\sin\theta} = \begin{bmatrix}
    0 & -1 & 0 \\
    1 & 0 & 0 \\
    0 & 0 & 0
  \end{bmatrix}
\end{gather}

\section{Hencky’s strain tensor}
The Hencky’s strain tensor can be calculated with the help of the left stretch tensor $\textbf{V}$ which is easy to determine with the help of deformation gradient and rotation tensor
\begin{gather}
  \textbf{V} = \textbf{F} ~ \textbf{R}^T  = \begin{bmatrix}
    0.7144 & 0.4139 & 0 \\
    0.4139 & 1.9687 & 0 \\
    0 & 0 & 1
  \end{bmatrix}.
\end{gather}
Hencky’s strain tensor is the natural logarithm of the left stretch tensor $\textbf{V}$
\begin{gather}
  \textbf{h} = \text{ln} \textbf{V} = \begin{bmatrix}
    -0.4227 & 0.3487 & 0 \\
    0.3487 & 0.6339 & 0 \\
    0 & 0 & 0
  \end{bmatrix}.
\end{gather}

%\section{Strain Measures and Deformation Analysis}
%
%\begin{minipage}[t]{0.45\textwidth}
%  \textbf{Green–Lagrange and Euler–Almansi strain tensor} \\[0.1 cm]
 % 
 % \begin{gather}
%    \textbf{E}(\textbf{X}) = \frac{1}{2} \left( \textbf{C} - \textbf{I} \right) = \frac{1}{2} \left( \textbf{F}^T \textbf{F} - \textbf{I} \right) \\
 %   \textbf{e}(\textbf{x}) = \frac{1}{2} \left( \textbf{I} - \textbf{c} \right) = \frac{1}{2} \left( \textbf{I} - \textbf{F}\textbf{F}^T \right)
%  \end{gather}
  
%  asdasd
%\end{minipage}
%\hspace{0.02\textwidth}
%\vline
%\hspace{0.02\textwidth}
%\begin{minipage}[t]{0.45\textwidth}
%  \textbf{Volume ratio, volume strain} \\[0.1 cm]
  
%  \begin{gather}
%    J = \det \textbf{F}(\textbf{X}_G) =  \\
%    \varepsilon_V = J - 1 = 
%  \end{gather}
%
%  asdasd
%\end{minipage}
\newpage
\label{page1}
\includepdf[pages=-]{figures/CM_I_HW.pdf}
\end{document}