\documentclass[12pt]{article}
\usepackage{blindtext}
\usepackage[top=25mm, bottom=25mm, left=20mm, right=20mm]{geometry}
\usepackage{t1enc}
\usepackage{fancyhdr}
\usepackage[english]{babel}
\usepackage{lastpage}
\usepackage{amsmath}
\usepackage{amssymb,physics}
\usepackage{amsthm}
\usepackage{empheq}
\usepackage{graphicx}
\usepackage{wrapfig}
\usepackage{tikz}
\usepackage{bm}
\usepackage{gensymb}
\usepackage{pdfpages}
\usepackage{hyperref}
\usepackage{multirow}
\usepackage{mdframed}
\usepackage[most]{tcolorbox}
\usepackage{pgfplots}
\usepackage{float}
\usepackage{caption}
\usepackage{subcaption}
\pgfplotsset{compat = newest}
\usepackage{listings}
\usepackage[makeroom]{cancel}
\usepackage{xcolor}

\definecolor{codegreen}{rgb}{0,0.6,0}
\definecolor{codegray}{rgb}{0.5,0.5,0.5}
\definecolor{codepurple}{rgb}{0.58,0,0.82}
\definecolor{backcolour}{rgb}{0.9059,0.902,0.9137}
\hypersetup{
    colorlinks=true,
    linkcolor=black,
    filecolor=magenta,      
    urlcolor=cyan,
}
\newtcolorbox{mygraybox}[1]{
  enhanced,
  colback = gray!10,
  colbacklower = black!80,
  colframe = black,
  rounded corners,
  boxrule = 0pt,
  leftrule = 2pt, % Adjust this for a thicker black frame
  toprule = 0pt, % Adjust this for a thicker black frame
  rightrule = 0pt, % Adjust this for a thicker black frame
  bottomrule = 0pt, % Adjust this for a thicker black frame
  left = 10pt,
  right = 10pt,
  top = 2pt,
  bottom = 2pt,
  arc = 1mm, % Adjust this for the roundness
  before upper = #1
}


\urlstyle{same}
\usepackage{graphicx}
\graphicspath{ {./images/} }
\usepackage{pict2e}
\usepackage{blindtext}
\usepackage{tikz}
\newcommand*\circledB{%
  \tikz[baseline=(char.base)]{%
    \node[shape=circle,draw,inner sep=1pt] (char) {b};%
  }%
}
% Table float box with bottom caption, box width adjusted to content


\newcommand*\circled[1]{%
  \tikz[baseline=(char.base)]{%
    \node[shape=circle,draw,inner sep=1pt] (char) {#1};%
  }%
}

\newcommand{\tens}[1]{%
  \mathbin{\mathop{\otimes}\limits_{#1}}%
}

\DeclareRobustCommand{\slashcirc}{{\mathpalette\doslashcirc\relax}}

\usepackage{graphicx,array}
\newcolumntype{C}[1]{>{\centering\let\newline\\\arraybackslash\hspace{130pt}}m{#1}}
\newcolumntype{L}[1]{>{\raggedright\let\newline\\\arraybackslash\hspace{130pt}}m{#1}}

\makeatother

\usepackage{amsmath}
\newcommand\aug{\fboxsep=-\fboxrule\!\!\!\fbox{\strut}\!\!\!}
\usepackage{wasysym}
\usepackage{tikz}
\usepackage{xcolor}
%\def\checkmark{\tikz\fill[scale=0.4](0,.35) -- (.25,0) -- (1,.7) -- (.25,.15) -- cycle;} 
\newcommand*\widefbox[1]{\fbox{\hspace{2em}#1\hspace{2em}}}
\newcommand{\fakesection}[1]{%
% Increase section counter
  % Add section mark (header)
  \addcontentsline{toc}{section}{\protect\numberline{\thesection}#1}% Add section to ToC
  % Add more content here, if needed.
}
\renewcommand{\headrulewidth}{0.5pt}
\renewcommand{\footrulewidth}{0.5pt}
\newcommand{\horrule}[1]{\rule{\linewidth}{#1}}
\pagestyle{fancy}
\newcommand\invisiblesection[1]{%
  \refstepcounter{section}%
  \addcontentsline{toc}{section}{\protect\numberline{}Feladatkitűzés}
  \sectionmark{1}}
\newcommand{\NEV}{Gergő Kelle}
\newcommand{\NEPTUN}{GBBNUL}
\newcommand{\TANTARGYKOD}{BMEGEMMNWCM}
\newcommand{\HFCIME}{Continuum Mechanics \\ II. Homework}
\newcommand{\HFFEJLECCIME}{Continuum Mechanics}
\title{\centering \includegraphics[width=10 cm , height=2.8cm]{figures/bme_logo_nagy.png}
\bigskip

\normalfont \normalsize \textsc{\centering Budapest University of Technology and Economics
\\ Faculty of Mechanical Engineering} \\ [12pt] \horrule{0.5pt} \\[0.4cm] \huge \HFCIME{} \\ \horrule{2pt} \\[0.5cm]}
\lhead{\TANTARGYKOD, \HFFEJLECCIME{}}
\rfoot{\thepage\ / \pageref{LastPage}}
\cfoot{}
\rhead{\NEV{}, \NEPTUN{}}
\author{\NEV{}}
\lstdefinestyle{mystyle}{
    backgroundcolor=\color{backcolour},   
    commentstyle=\color{codegreen},
    keywordstyle=\color{magenta},
    numberstyle=\tiny\color{codegray},
    stringstyle=\color{codepurple},
    basicstyle=\ttfamily\footnotesize,
    breakatwhitespace=false,         
    breaklines=true,                 
    captionpos=b,                    
    keepspaces=true,                 
    numbers=left,                    
    numbersep=5pt,                  
    showspaces=false,                
    showstringspaces=false,
    showtabs=false,                  
    tabsize=2
}

\lstset{style=mystyle}

\begin{document}
\includepdf[pages=1]{data/contmech_hw2.pdf}
\pagebreak
\maketitle

\pagebreak
\tableofcontents
\newpage
\section{Task - Experimental data}
In this task I've created a visual representation of the given dataset of Table \ref{tab:mechanical_tests}.

\begin{table}[htbp]
    \centering
    \caption{Mechanical Test Results}
    \label{tab:mechanical_tests}
    \begin{tabular}{cccc}
        & \textbf{Uniaxial Test} & \textbf{Equibiaxial Test} & \textbf{Planar Test} \\[0.2 cm]
        Eng. Strain [\%] & Eng. Stress [MPa] & Eng. Stress [MPa] & Eng. Stress [MPa] \\ [0.1 cm] \hline
        0 & 0 & 0 & 0 \\
        30 & 8 & 18 & 12 \\
        60 & 15 & 30 & 18 \\
        90 & 20 & 47 & 26 \\
        120 & 28 & 59 & 35 \\
        150 & 43 & 85 & 50 \\
        180 & 65 & 105 & 69 \\
        210 & 92 & 140 & 95 \\
        240 & 116 & 176 & 132 \\
        270 & 171 & 231 & 169 \\
        300 & 220 & 307 & 227
    \end{tabular}
\end{table}

\noindent Before the visualization I've calculated stretch from engineering strain with the following equation

\begin{equation}
  \lambda = \dfrac{\varepsilon_{eng}}{100} + 1 ~.
\end{equation}

\noindent The resulting solution is plotted at Figure \ref{fig:plot1}.

\begin{figure}[ht!]
  \begin{center}
  \includegraphics[width=0.7\textwidth]{figures/edited/plot_task1.pdf}
  \captionof{figure}{Stress-Stretch Curves for Different Tests}
  \label{fig:plot1}
  \end{center}
\end{figure}

\section{Task - Model fitting to the uniaxial data}
As the Homework determined the quality function that is used to describe the parameter fitting correctness is the root mean squared relative error which should be implemented. In general it has the following form:
\begin{equation}
  Q = \sqrt{\dfrac{1}{n} \sum_{i=1}^{n} \left( \dfrac{P_i^{exp} - P_i^{sim}}{P_i^{exp}}\right) ^2 }
\end{equation}
Where $P^{sim}$ simulated nominal stress is determined based on the type of the experimental test. Therefore the different nominal stresses are calculated the following ways:
\begin{gather}
  P_U = \sum_{i=1}^{2} \dfrac{2\mu_i}{\alpha_i} \left( \lambda^{\alpha_i - 1} - \lambda ^{-0.5\alpha_i-1}  \right) \\
  P_B = \sum_{i=1}^{2} \dfrac{2\mu_i}{\alpha_i} \left( \lambda^{\alpha_i - 1} - \lambda ^{-2\alpha_i-1}  \right) \\
  P_P = \sum_{i=1}^{2} \dfrac{2\mu_i}{\alpha_i} \left( \lambda^{\alpha_i - 1} - \lambda ^{-\alpha_i-1}  \right)
\end{gather}

\noindent As for the model fitting I've used only the uniaxial data in this task. To calculate the parameters of $\mu_1, \mu_2, \alpha_1, \alpha_2$ I've used \textit{Python} programming language in which I've used the \textit{minimize} from \textit{scipy.optimize}. For the best solution I've tried different methods simultaniously to achieve the best quality result for the paramters. The resulting parameters are showed in Table \ref{tab:quality_results_01}.

\begin{table}[htbp]
  \centering
  \caption{Resulting Parameters for Uniaxial Data Model Fitting}
  \label{tab:quality_results_01}
  \begin{tabular}{cc}
      \multicolumn{2}{c}{\textbf{Material Parameters}} \\ \hline
      $\mu_1$ & 10.5544 \\
      $\mu_2$ & 2.1012 \\
      $\alpha_1$ & -0.7732 \\
      $\alpha_2$ & 4.9987 \\
  \end{tabular}
\end{table}

\noindent The visualization of the solution is displayed in Figure \ref{fig:plot1_01}.

\begin{figure}[ht!]
  \begin{center}
  \includegraphics[width=0.7\textwidth]{figures/edited/uniaxial_plot.pdf}
  \captionof{figure}{Uniaxial-Only Model and Uniaxial Data Set Comparison}
  \label{fig:plot1_01}
  \end{center}
\end{figure}

\newpage
\section{Task - Quality in case of uniaxial data model fitting}
In the second task, parameter selection was optimized to minimize the uniaxial quality function ($Q_U$), leading to the exclusion of other datasets. Consequently, higher anticipated values for $Q_B$ and $Q_P$ were expected. However, unexpectedly, the planar quality function exhibited a superior result compared to the uniaxial.

\begin{table}[htbp]
  \centering
  \caption{Quality Function Results for uniaxial data model fitted parameters}
  \label{tab:quality_results}
  \begin{tabular}{cccc}
      & \textbf{Uniaxial} & \textbf{Equibiaxial} & \textbf{Planar}\\[0.2 cm] \hline
      Quality [\%] & 4.1048 & 9.4568 & 3.3923 
  \end{tabular}
\end{table}


\section{Task - Model accuracy and conclusions}
In general, it can be stated that the optimally tuned parameters based solely on uniaxial measurement results are insufficient to cover the values of engineering stresses in case of equibiaxial or planar extensions. Consequently, a quality value below 5\% cannot be guaranteed for the model. To achieve improved results, it is more prudent to fit the parameters $\mu_1, \mu_2, \alpha_1, \alpha_2$ based on the entire set of measurement data which will be calculated and proved later on. 
\medskip

\noindent Figure \ref{fig:plot5} and \ref{fig:plot6} visualize the resulting solution. 

\begin{figure}[ht!]
  \begin{center}
  \includegraphics[width=0.7\textwidth]{figures/edited/equibiaxial_plot.pdf}
  \captionof{figure}{Uniaxial-Only Model and Equibiaxial Data Set Comparison}
  \label{fig:plot5}
  \end{center}
\end{figure}

\begin{figure}[ht!]
  \begin{center}
  \includegraphics[width=0.7\textwidth]{figures/edited/planar_plot.pdf}
  \captionof{figure}{Uniaxial-Only Model and Planar Data Set Comparison}
  \label{fig:plot6}
  \end{center}
\end{figure}

\newpage
\section{Task - Model Fitting to All Experimental Data}
When fitting to all experimental - measured data a summarization of quality function should be used. The new used quality function value is the mean of the uniaxial, quibiaxial and planar quality functiuns value.

\begin{equation}
  Q = \frac{Q_U +Q_B + Q_P}{3}
\end{equation}

\noindent Running the same algorith that was used before on all data resulting different parameter set that is listed in Table \ref{tab:param_all}.

\begin{table}[htbp]
  \centering
  \caption{Resulting Parameters for All Data Model Fitting}
  \label{tab:param_all}
  \begin{tabular}{ccc}
      \multicolumn{2}{c}{\textbf{Material Parameters}} & Difference [\%] \\ \hline
      $\mu_1$ & 11.0175 & 4.39 \\
      $\mu_2$ & 1.9718 & -6.16\\
      $\alpha_1$ & -0.9609 & 24.28\\
      $\alpha_2$ & 5.0453 & 0.93\\
  \end{tabular}
\end{table}
It is intresting to point out that the parameters (on average) for fitting to every data is changed only for 8.94 \% compared to the Only-Uniaxial Model.

\begin{table}[htbp]
  \centering
  \caption{Quality Function Results to All Data Model Fitted Parameters}
  \label{tab:quality_results_total}
  \begin{tabular}{ccccc}
      & \textbf{All Data} & \textbf{Uniaxial} & \textbf{Equibiaxial} & \textbf{Planar}\\[0.1 cm] \hline
      Quality [\%] & 3.4626 & 4.2447 & 3.0041 & 3.1389 
  \end{tabular}
\end{table}

\section{Task - All Data Fitted Model Accuracy and Conclusions}
The accuracy can be described the best with the mean quality function value which is 3.4626 \%, in comparison when fitting only to the uniaxial extension engineer stress data a mean quality function value of 5.6513 \% was achived. This shows that unsuprisingly fitting to all data is more accurate when speaking of the whole material model.
\medskip

The visualization of the solution is displayed in Figure \ref{fig:plot11} - \ref{fig:plot13}.

\begin{figure}[ht!]
  \begin{center}
  \includegraphics[width=0.7\textwidth]{figures/edited/uniaxial_plot_total.pdf}
  \captionof{figure}{Comprehensive Model and Uniaxial Data Set Comparison}
  \label{fig:plot11}
  \end{center}
\end{figure}

\begin{figure}[ht!]
  \begin{center}
  \includegraphics[width=0.7\textwidth]{figures/edited/equibiaxial_plot_total.pdf}
  \captionof{figure}{Comprehensive Model and Equibiaxial Data Set Comparison}
  \label{fig:plot12}
  \end{center}
\end{figure}

\begin{figure}[ht!]
  \begin{center}
  \includegraphics[width=0.7\textwidth]{figures/edited/planar_plot_total.pdf}
  \captionof{figure}{Comprehensive Model and Planar Data Set Comparison}
  \label{fig:plot13}
  \end{center}
\end{figure}

\newpage
\section{Task - Calculation of the unknown scalar $k$}

\noindent In this task I need to determine the unknown scalar $k$ found in the given displacement field $\textbf{U}$. Since the homework is about an incompressible rubber-like material thus the volume ratio $J=1$ therefore $k$ can be obtained if the deformation gradient ($\textbf{F}$) is known. \\

Based on my data the displacement field $\textbf{U}$ is the following:

\begin{equation}
  \textbf{U} = \begin{bmatrix}
    2 - k \cdot Y \\
    -0.6 X + 0.1 Y \\
    -2
  \end{bmatrix}
\end{equation}

The deformation gradient ($\textbf{F}$) is determined as

\begin{equation}
  \textbf{F} = \text{Grad } \textbf{U} + \textbf{I} =     \begin{bmatrix}
    \dfrac{\partial \textbf{U}_{1}}{\partial X} & \dfrac{\partial \textbf{U}_{1}}{\partial Y} & \dfrac{\partial \textbf{U}_{1}}{\partial Z} \\[0.4 cm]
    \dfrac{\partial \textbf{U}_{2}}{\partial X} & \dfrac{\partial \textbf{U}_{2}}{\partial Y} & \dfrac{\partial \textbf{U}_{2}}{\partial Z} \\[0.4 cm]
    \dfrac{\partial \textbf{U}_{3}}{\partial X} & \dfrac{\partial \textbf{U}_{3}}{\partial Y} & \dfrac{\partial \textbf{U}_{3}}{\partial Z} \\
\end{bmatrix}
+     \begin{bmatrix}
  1 & 0 & 0\\
  0 & 1 & 0\\
  0 & 0 & 1\\
\end{bmatrix} = \begin{bmatrix}
  1 & -k & 0\\
  -0.6 & 1.1 & 0 \\
  0 & 0 & 1 \\
\end{bmatrix}.
\end{equation}

\begin{gather}
  J = 1 = \det \textbf{F}
\end{gather}
The resulting equation can be solved for $k$ unknown scalar since it is the only unkown value.
\begin{equation}
  k = 0.1667
\end{equation}
Substituting the $k$ value into the deformation gradient gives it's numerical value of
\begin{equation}
  \textbf{F} = \displaystyle \left[\begin{matrix}1 & -0.1667 & 0\\-0.6 & 1.1 & 0\\0 & 0 & 1\end{matrix}\right].
\end{equation}

\section{Task - Determine the principal stretches}
In order to determine the principal stretches and the corresponding principal Eulerian directions obtaining the left Cauchy-Green deformation tensor is the efficient way (since we know the deformation tensor from the task before).

\begin{equation}
  \textbf{b} = \textbf{F} \cdot \textbf{F}^T = \displaystyle \left[\begin{matrix}1.028 & -0.7833 & 0\\-0.7833 & 1.57 & 0\\0 & 0 & 1.0\end{matrix}\right]
\end{equation}

\begin{table}[htbp]
  \centering
  \renewcommand{\arraystretch}{1.1} % Adjust the value to increase or decrease the vertical spacing
  \caption{Principal Stretch and Eulerian Directions}
  \label{tab:eig}
  \begin{tabular}{ccc}
      $i$ & Principal Stretch ($\lambda_i$) & Principal Eulerian Direction ($\textbf{n}_i ^T$) \\ \hline
      1 & 0.6855 & $\begin{bmatrix}
        -0.8146 & -0.5801 & 0
      \end{bmatrix}$\\[0.1 cm]
      2 & 1.4587 & $\begin{bmatrix}
        0.5801 & -0.8146 & 0
      \end{bmatrix}$\\[0.1 cm]
      3 & 1 & $\begin{bmatrix}
        0 & 0 & 1
      \end{bmatrix}$\\
  \end{tabular}
\end{table}

\section{Task - Cauchy stress and Mises Equivalent Stress}
In this task I'll use the parameters determined in the fifth Task which is present in the Table \ref{tab:param_all}. According to the homework description the used strain potential function ($W$) is the 2$^{\text{nd}}$-order incompressible Ogden's isotropic hyperelastic constitutive model which is given in the following form:
\begin{equation}
  W = \sum_{k = 1}^{2} \dfrac{2 \mu_k}{\alpha_k^2} \left( \lambda_1^{\alpha_k} + \lambda_2^{\alpha_k} + \lambda_3^{\alpha_k} - 3 \right).
\end{equation}

\noindent Cauchy stress tensor ($\bm{\sigma}$) can be determined with the help of the strain potential function and the Left Cauchy-Green def. tensor (\textbf{b}).
\begin{equation}
  \bm{\sigma} = \dfrac{2}{J} \left[ \left( \dfrac{\partial W}{\partial I_1} + I_1 \dfrac{\partial W}{\partial I_2} \right) \textbf{b} - \dfrac{\partial W}{\partial I_2} \textbf{b}^2 + I_3 \dfrac{\partial W}{\partial \lambda_3} \textbf{I} \right]
\end{equation}

\noindent As for the partial derivatives of the strain potential function the following equations can be obtained:

\begin{gather}
  \dfrac{\partial W}{\partial \lambda_1} = 2 \left( \dfrac{\mu_1}{\alpha_1} \lambda_1 ^{\alpha_1-1} + \dfrac{\mu_2}{\alpha_2} \lambda_1^{\alpha_2 - 1} \right), \\
  \dfrac{\partial W}{\partial \lambda_2} = 2 \left( \dfrac{\mu_1}{\alpha_1} \lambda_2 ^{\alpha_1-1} + \dfrac{\mu_2}{\alpha_2} \lambda_2^{\alpha_2 - 1} \right), \\
  \dfrac{\partial W}{\partial \lambda_3} = 2 \left( \dfrac{\mu_1}{\alpha_1} \lambda_3 ^{\alpha_1-1} + \dfrac{\mu_2}{\alpha_2} \lambda_3^{\alpha_2 - 1} \right).
\end{gather}

\noindent Since we are talking about incompressible isotropic case $J = 1$ and $I_3 = 1$ thus
\begin{equation}
  W = W \left( I_1, I_2, I_3 \right) \quad \rightarrow \quad W = W \left(I_1, I_2 \right)
\end{equation}
In this case only the deviatoric stress can be calculated from the constitutive equation.

\begin{gather}
  \bm{\sigma} = \text{dev} \left[ \bm{\sigma} \right] + \text{sph} \left[ \bm{\sigma} \right] = \textbf{s} + \textbf{p} \\
  \textbf{s} = \text{dev} \left[ \sum_{i=1}^{3} \lambda_i \dfrac{\partial W}{\partial \lambda_i} \textbf{n}_i \tens{} \textbf{n}_i \right] = \displaystyle \left[\begin{matrix}-3.4953 & -10.4615 & 0\\-10.4615 & 3.7462 & 0\\0 & 0 & -0.2509\end{matrix}\right]~ \text{[MPa]}
\end{gather}

\noindent In the context of examining a plain stress state, the determination of the pressure field (\textbf{p}) is achieved through the imposition of a boundary condition, specifically requiring the stress in the z-direction to be zero $\left( \sigma_z = \tau_{xz} = \tau_{yz} = 0 \right)$, in accordance with the planar stress nature of the problem.

\begin{equation}
  \textbf{p} = - \textbf{s}_{33} ~ \textbf{I} = \displaystyle \left[\begin{matrix}0.2509 & 0 & 0\\0 & 0.2509 & 0\\0 & 0 & 0.2509\end{matrix}\right]  ~ \text{[MPa]}
\end{equation}

\noindent Since we know the deviatoric and the spherical part of the tensor, the Cauchy stress tensor can be obtained as

\begin{equation}
  \bm{\sigma} = \textbf{s} + \textbf{p}  = \displaystyle \left[\begin{matrix}-3.2443 & -10.4615 & 0\\ -10.4615 & 3.9972 & 0\\0 & 0 & 0\end{matrix}\right]  ~ \text{[MPa]}.
\end{equation}

\noindent Regarding the Mises equivalent stress, with the help of the deviatoric part of the Cauchy stress tensor it can be calculated as follows:

\begin{equation}
  \sigma_{\text{VM}} = \sqrt{\frac{3}{2} \tr \left( \textbf{s} \tens{} \textbf{s} \right) } = 19.1782 ~ \text{[MPa]}.
\end{equation}

\newpage
\label{page1}
\includepdf[pages=1, pagecommand={\pagestyle{empty}\section{Appendix}
\thispagestyle{empty}}, offset=0mm -10mm]{figures/Continuum Mechanics 2 HW.pdf}
\includepdf[pages=2-]{figures/Continuum Mechanics 2 HW.pdf}
\end{document}